\documentclass[12pt]{book}
\usepackage[width=4.375in, height=7.0in, top=1.0in, papersize={5.5in,8.5in}]{geometry}
\usepackage[pdftex]{graphicx}
\usepackage{amsmath}
\usepackage{amssymb}
\usepackage{tipa}
\usepackage{textcomp}
\usepackage{fancyhdr}

%Handling font switching
\usepackage{tgbonum}
\usepackage[T1]{fontenc}

%Fonts, including a handwritten-looking one
\usepackage{emerald}
\usepackage[T1]{fontenc}
%%Another font I had considered using
%\DeclareTextFontCommand{\texthw}{\ECFTeenSpirit}
\DeclareTextFontCommand{\texthw}{\ECFAugie}
\hyphenation{For-ster}

\pagestyle{fancy}
\renewcommand{\chaptermark}[1]{\markboth{#1}{}}
\renewcommand{\sectionmark}[1]{\markright{\thesection\ #1}}
\fancyhf{}
\fancyhead[LE,RO]{\bfseries\thepage}
\fancyhead[LO]{\bfseries\rightmark}
\fancyhead[RE]{\bfseries\leftmark}
\renewcommand{\headrulewidth}{0.5pt}
\renewcommand{\footrulewidth}{0pt}
\addtolength{\headheight}{0.5pt}
\setlength{\footskip}{0in}
\renewcommand{\footruleskip}{0pt}
\fancypagestyle{plain}{%
\fancyhead{}
\renewcommand{\headrulewidth}{0pt}
}
%
%\parindent 0in
\parskip 0.06in
%
\begin{document}
\frontmatter
%

``The Machine Stops" was first published in the \textit{Oxford and Cambridge Review} in 1909. It is in the public domain in the United States because it was published before January 1, 1923.
The author died in 1979, so this work is also in the public domain in countries and areas where the copyright term is the author's life plus 30 years or less. This work may also be in the public domain in countries and areas with longer native copyright terms that apply the rule of the shorter term to foreign works.

``Scanners Live in Vain" was first published in \textit{ Fantasy Book} in 1950. Publicly available records show no renewal of copyright. The author died in 1966, so this work is also in the public domain in countries and areas where the copyright term is the author's life plus 30 years or less. This work may also be in the public domain in countries and areas with longer native copyright terms that apply the rule of the shorter term to foreign works.
%http://en.wikipedia.org/wiki/Wikipedia:Reference_desk/Archives/Humanities/2007_April_15
%I'm checking for a copyright renewal on the short story "Scanners Live in Vain", by Cordwainer Smith. It was first published in a magazine, Fantasy Book, in 1950; the copyright on that magazine was not renewed. The copyright on the short story itself was not individually renewed, and it doesn't appear in any copyright renewal records with Smith's pseudonym or real name in them.--Grendelkhan, http://en.wikipedia.org/wiki/User:Grendelkhan
%

This selection, arrangement, and the foreword are licensed under the Creative Commons At\-trib\-ution-\newline{}Share\-Alike 3.0 Unported License. To view a copy of this license, visit http://creativecommons.org/licenses/by-sa/3.0/ or send a letter to Creative Commons, 444 Castro Street, Suite 900, Mountain View, California, 94041, USA.
\tableofcontents
%
\chapter{Foreword}

February 2014

I wanted to share two pieces of old-fashioned science fiction in this collection. Both of these works have influenced several generations of speculative fiction writers, and both of them are good stories that have stood the test of time. Both also have the sort of magic it would take to reveal water to a fish: because they were written before I was born, about a time long after my death, they have revealed to me a lot more about my own culture and assumptions than other sorts of writing seem able to. I hope they do the same for the reader.

The two stories in question are  \emph{The Machine Stops} and \emph{Scanners Live In Vain}. I wouldn't quite call either ``dystopian'', but both explore ideas of powerful systems in need of correction.

Each time I read \emph{The Machine Stops}, I see yet another idea or circumstance that is shockingly contemporary for content dating back to 1909. The language is definitely not from the 21\textsuperscript{st} Century, but the lifestyle depicted---a routine of calling up interesting videos and music, mining older cultures for ideas, and pouring it all into remote discussion with ``friends'' one never sees in person, while forgotten automatic systems take care of the necessities of life---looks more current with each passing year. 

But I suspect I glamorize E. M. Forster's clear view of the future, due to a distorted view the past: with apologies to William Gibson, I forget how much of the present was already there---it was just unevenly distributed. The author lived in a time of powered flight, cinematography, and various methods for audio storage and transmission. Dirigibles, a global trade network, and the telegraph system had already resulted in an in\-ter-con\-nect\-ed world. While pervasive access to the intenet would have to wait for at least four score years, the concepts of digitization, error correction, automated repeating, and message switching (not entirely unlike packet switching) had already been commercialized and broadly implemented: global telegraph networks have many of the same needs as global computer networks. Whether Forster was aware of it or not, raster images had also been transmitted electronically, using a device called a Telediagraph. While there weren't TED talks in 1909, there was the Chautauquah circuit, which had similar cultural mechanics even though it relied on physical travel.

This is not to diminish the deft choices of extrapolation that led to ideas like video conferencing and von Neuman-style clanking replicators. But the best science fiction is less an effort to predict the future, than to offer new perspectives on the present. The Machine Stops may resonate with nightmare scenarios about a collapse of global commerce or the de-stabilization of US hegemony, but only because it deals with timeless themes. I would guess that the author examined these themes because they were also apparent in the trade and military empire that had filled the Earth and subdued it in his own time: Edwardian England.

I don't hold romantic ideas about the eternal rightness of some authentic natural way, but studying science and practicing engineering has instilled deep respect, even awe, for the capacity of living systems to heal and to adapt. The systems we build can hold our purposes and unflinchingly drive continued change, but on some level, a lack of adaptability will cause problems when our contraptions encounter limits. 

I can think of several Machines that seem both totalizing and brittle as of this writing, and several confraternities of Scanners who are too absorbed in adjusting the dials and maintaining the \textit{status quo ante} to see the larger picture; I'm sure the reader can, too. What has connected these two stories to my present experience most strongly, though, has been the recent behavior of our intellectual property system. The US Supreme Court ruled that genes cannot be patented, but farmers are still being sued for allowing proprietary pollen to blow into their fields. Patent trolls continue to make good money by stifling innovation. The sharing and adaptation of music, which I think might be even more fundamentally human than inventing new tools or growing food, is being encumbered and co-opted. And worst of all, the ``mending apparatus'' of publishing---journalism---now seems diminished in its capacity to correct itself.

The future of \emph{Scanners Live In Vain} is several decades younger, but its scenes middle-class domesticity served by a whiz-bang transit system sound a lot more old-fashioned. And while both stories include some frank orientalism, Cordwainer Smith shows more mid-century narrow-mindedness in his confident assertions of various sorts of privilege. It isn't much for accurate prediction, but I was just as happy to include it.

I can't think of much from this latter story that came true in the real world, but I see large parallels with other fictional works: notably, The Borg, from \emph{Star Trek}, and the arc of the title character (humoring SCOTUS in their assertion that corporations are persons) of \emph{Monsters Incorporated}. Smith gives us more of an insider's perspective on the sacrifices necessary to keep large systems working, and a more realistic portrayal of the very human work of making such hard decisions. 

Subchiefs and Magnates of the Instrumentality don't have to work as hard as denizens of The Machine to un-do their alienation: the process of re-connecting is mechanized, too. But I find Smith's picture of the process to be more realistic about the internal dynamics: we need social and emotional support to even imagine a different way of life, let alone find the courage to attempt it.

Part of what attracted me to these two stories is their public domain status. The fact that no one can use the legal system to stop anyone from publishing them inspires doubt that publishing them could be profitable, and causes the monetized part of our culture to pay them less attention. I have edited both works to correct what I presume are character-encoding and OCR errors, but it should be noted that I also added chapter names to what had originally been numbered sections in Cordwainer Smith's work.

These two pieces of writing also speak to my struggle to adapt, a struggle I imagine most living things share. Both remind me to pay attention to my gut feelings, and to stay in touch with people who are aware and adaptable. The endings are very different, but I find both to be inspiring. I hope they can have a similar impact, more broadly: Living things heal, and our living traditions are stronger for the criticism of speculative fiction like this.

As Forster wrote elsewhere, ``Only connect!''

Joel Hollingsworth

%
\mainmatter
%
\chapter{The Machine Stops}
\thispagestyle{empty}
%{\hspace{0.25in} \includegraphics{./ru_sun.jpg} }
\section*{\huge \center E. M. Forster}
\newpage
%
%THE MACHINE STOPS


%The ''Machine Stops'' was first published in the Oxford and Cambridge Review in 1909
%Copyright \copyright 1947 E.M. Forster
%downloaded Dec. 2013 from http://archive.ncsa.illinois.edu/prajlich/forster.html

\section{The Air-Ship}

Imagine, if you can, a small room, hexagonal in shape, like the cell of a bee. It is lighted neither by window nor by lamp, yet it is filled with a soft radiance. There are no apertures for ventilation, yet the air is fresh. There are no musical instruments, and yet, at the moment that my meditation opens, this room is throbbing with melodious sounds. An armchair is in the centre, by its side a reading-desk---that is all the furniture. And in the armchair there sits a swaddled lump of flesh---a woman, about five feet high, with a face as white as a fungus. It is to her that the little room belongs.

An electric bell rang.

The woman touched a switch and the music was silent.

``I suppose I must see who it is'', she thought, and set her chair in motion. The chair, like the music, was worked by machinery and it rolled her to the other side of the room where the bell still rang importunately.

``Who is it?'' she called. Her voice was irritable, for she had been interrupted often since the music began. She knew several thousand people, in certain directions human intercourse had advanced enormously.

But when she listened into the receiver, her white face wrinkled into smiles, and she said:

``Very well. Let us talk, I will isolate myself. I do not expect anything important will happen for the next five minutes---for I can give you fully five minutes, Kuno. Then I must deliver my lecture on \emph{Music during the Australian Period}.''

She touched the isolation knob, so that no one else could speak to her. Then she touched the lighting apparatus, and the little room was plunged into darkness.

``Be quick!'' She called, her irritation returning. ``Be quick, Kuno; here I am in the dark wasting my time.''

But it was fully fifteen seconds before the round plate that she held in her hands began to glow. A faint blue light shot across it, darkening to purple, and presently she could see the image of her son, who lived on the other side of the earth, and he could see her.

``Kuno, how slow you are.''

He smiled gravely.

``I really believe you enjoy dawdling.''

``I have called you before, mother, but you were always busy or isolated. I have something particular to say.''

``What is it, dearest boy? Be quick. Why could you not send it by pneumatic post?''

``Because I prefer saying such a thing. I want---''

``Well?''

``I want you to come and see me.''

Vashti watched his face in the blue plate.

``But I can see you!'' she exclaimed. ``What more do you want?''

``I want to see you not through the Machine,'' said Kuno. ``I want to speak to you not through the wearisome Machine.''

``Oh, hush!'' said his mother, vaguely shocked. ``You mustn't say anything against the Machine.''

``Why not?''

``One mustn't.''

``You talk as if a god had made the Machine,'' cried the other.

``I believe that you pray to it when you are unhappy. Men made it, do not forget that. Great men, but men. The Machine is much, but it is not everything. I see something like you in this plate, but I do not see you. I hear something like you through this telephone, but I do not hear you. That is why I want you to come. Pay me a visit, so that we can meet face to face, and talk about the hopes that are in my mind.''

She replied that she could scarcely spare the time for a visit.

``The air-ship barely takes two days to fly between me and you.''

``I dislike air-ships.''

``Why?''

``I dislike seeing the horrible brown earth, and the sea, and the stars when it is dark. I get no ideas in an air-ship.''

``I do not get them anywhere else.''

``What kind of ideas can the air give you?''

He paused for an instant.

``Do you not know four big stars that form an oblong, and three stars close together in the middle of the oblong, and hanging from these stars, three other stars?''

``No, I do not. I dislike the stars. But did they give you an idea? How interesting; tell me.''

``I had an idea that they were like a man.''

``I do not understand.''

``The four big stars are the man's shoulders and his knees. The three stars in the middle are like the belts that men wore once, and the three stars hanging are like a sword.''

``A sword?'' %internet had a semicolon between the question mark & quotation mark, here. ?;

``Men carried swords about with them, to kill animals and other men.''

``It does not strike me as a very good idea, but it is certainly original. When did it come to you first?''

``In the air-ship---'' He broke off, and she fancied that he looked sad. She could not be sure, for the Machine did not transmit nuances of expression. It only gave a general idea of people---an idea that was good enough for all practical purposes, Vashti thought. The imponderable bloom, declared by a discredited philosophy to be the actual essence of intercourse, was rightly ignored by the Machine, just as the imponderable bloom of the grape was ignored by the manufacturers of artificial fruit. Something ``good enough'' had long since been accepted by our race.

``The truth is,'' he continued, ``that I want to see these stars again. They are curious stars. I want to see them not from the air-ship, but from the surface of the earth, as our ancestors did, thousands of years ago. I want to visit the surface of the earth.''

She was shocked again.

``Mother, you must come, if only to explain to me what is the harm of visiting the surface of the earth.''

``No harm,'' she replied, controlling herself. ``But no advantage. The surface of the earth is only dust and mud, no advantage. The surface of the earth is only dust and mud, no life remains on it, and you would need a respirator, or the cold of the outer air would kill you. One dies immediately in the outer air.''

``I know; of course I shall take all precautions.''

``And besides---''

``Well?''

She considered, and chose her words with care. Her son had a queer temper, and she wished to dissuade him from the expedition.

``It is contrary to the spirit of the age,'' she asserted.

``Do you mean by that, contrary to the Machine?''

``In a sense, but---''

His image is the blue plate faded.

``Kuno!''

He had isolated himself.

For a moment Vashti felt lonely.

Then she generated the light, and the sight of her room, flooded with radiance and studded with electric buttons, revived her. There were buttons and switches everywhere---buttons to call for food, for music, for clothing. There was the hot-bath button, by pressure of which a basin of (imitation) marble rose out of the floor, filled to the brim with a warm deodorized liquid. There was the cold-bath button. There was the button that produced literature. And there were of course the buttons by which she communicated with her friends. The room, though it contained nothing, was in touch with all that she cared for in the world.

Vashanti's next move was to turn off the isolation switch, and all the accumulations of the last three minutes burst upon her. The room was filled with the noise of bells, and speaking-tubes. What was the new food like? Could she recommend it? Has she had any ideas lately? Might one tell her one's own ideas? Would she make an engagement to visit the public nurseries at an early date?---say this day month.

To most of these questions she replied with irritation---a growing quality in that accelerated age. She said that the new food was horrible. That she could not visit the public nurseries through press of engagements. That she had no ideas of her own but had just been told one---that four stars and three in the middle were like a man: she doubted there was much in it. Then she switched off her correspondents, for it was time to deliver her lecture on Australian music.

The clumsy system of public gatherings had been long since abandoned; neither Vashti nor her audience stirred from their rooms. Seated in her armchair she spoke, while they in their armchairs heard her, fairly well, and saw her, fairly well. She opened with a humorous account of music in the pre-Mongolian epoch, and went on to describe the great outburst of song that followed the Chinese conquest. Remote and prim\ae val as were the methods of I-San-So and the Brisbane school, she yet felt (she said) that study of them might repay the musicians of today: they had freshness; they had, above all, ideas. Her lecture, which lasted ten minutes, was well received, and at its conclusion she and many of her audience listened to a lecture on the sea; there were ideas to be got from the sea; the speaker had donned a respirator and visited it lately. Then she fed, talked to many friends, had a bath, talked again, and summoned her bed.

The bed was not to her liking. It was too large, and she had a feeling for a small bed. Complaint was useless, for beds were of the same dimension all over the world, and to have had an alternative size would have involved vast alterations in the Machine. Vashti isolated herself---it was necessary, for neither day nor night existed under the ground---and reviewed all that had happened since she had summoned the bed last. Ideas? Scarcely any. Events---was Kuno's invitation an event?

By her side, on the little reading-desk, was a survival from the ages of litter---one book. This was the Book of the Machine. In it were instructions against every possible contingency. If she was hot or cold or dyspeptic or at a loss for a word, she went to the book, and it told her which button to press. The Central Committee published it. In accordance with a growing habit, it was richly bound.

Sitting up in the bed, she took it reverently in her hands. She glanced round the glowing room as if some one might be watching her. Then, half ashamed, half joyful, she murmured ``O Machine!'' and raised the volume to her lips. Thrice she kissed it, thrice inclined her head, thrice she felt the delirium of acquiescence. Her ritual performed, she turned to page 1367, which gave the times of the departure of the air-ships from the island in the southern hemisphere, under whose soil she lived, to the island in the northern hemisphere, whereunder lived her son.

She thought, ``I have not the time.''

She made the room dark and slept; she awoke and made the room light; she ate and exchanged ideas with her friends, and listened to music and attended lectures; she make the room dark and slept. Above her, beneath her, and around her, the Machine hummed eternally; she did not notice the noise, for she had been born with it in her ears. The earth, carrying her, hummed as it sped through silence, turning her now to the invisible sun, now to the invisible stars. She awoke and made the room light.

``Kuno!''

``I will not talk to you.'' he answered, ``until you come.''

``Have you been on the surface of the earth since we spoke last?''

His image faded.

Again she consulted the book. She became very nervous and lay back in her chair palpitating. Think of her as without teeth or hair. Presently she directed the chair to the wall, and pressed an unfamiliar button. The wall swung apart slowly. Through the opening she saw a tunnel that curved slightly, so that its goal was not visible. Should she go to see her son, here was the beginning of the journey.

Of course she knew all about the communication-system. There was nothing mysterious in it. She would summon a car and it would fly with her down the tunnel until it reached the lift that communicated with the air-ship station: the system had been in use for many, many years, long before the universal establishment of the Machine. And of course she had studied the civilization that had immediately preceded her own---the civilization that had mistaken the functions of the system, and had used it for bringing people to things, instead of for bringing things to people. Those funny old days, when men went for change of air instead of changing the air in their rooms! And yet---she was frightened of the tunnel: she had not seen it since her last child was born. It curved---but not quite as she remembered; it was brilliant---but not quite as brilliant as a lecturer had suggested. Vashti was seized with the terrors of direct experience. She shrank back into the room, and the wall closed up again.

``Kuno,'' she said, ``I cannot come to see you. I am not well.''

Immediately an enormous apparatus fell on to her out of the ceiling, a thermometer was automatically laid upon her heart. She lay powerless. Cool pads soothed her forehead. Kuno had telegraphed to her doctor.

So the human passions still blundered up and down in the Machine. Vashti drank the medicine that the doctor projected into her mouth, and the machinery retired into the ceiling. The voice of Kuno was heard asking how she felt.

``Better.'' Then with irritation: ``But why do you not come to me instead?''

``Because I cannot leave this place.''

``Why?''

``Because, any moment, something tremendous many happen.''

``Have you been on the surface of the earth yet?''

``Not yet.''

``Then what is it?''

``I will not tell you through the Machine.''

She resumed her life.

But she thought of Kuno as a baby, his birth, his removal to the public nurseries, her own visit to him there, his visits to her---visits which stopped when the Machine had assigned him a room on the other side of the earth. ``Parents, duties of,'' said the book of the Machine, ``cease at the moment of birth. P.422327483.'' True, but there was something special about Kuno---indeed there had been something special about all her children---and, after all, she must brave the journey if he desired it. And ``something tremendous might happen''. What did that mean? The nonsense of a youthful man, no doubt, but she must go. Again she pressed the unfamiliar button, again the wall swung back, and she saw the tunnel that curves out of sight. Clasping the Book, she rose, tottered on to the platform, and summoned the car. Her room closed behind her: the journey to the northern hemisphere had begun.

Of course it was perfectly easy. The car approached and in it she found armchairs exactly like her own. When she signaled, it stopped, and she tottered into the lift. One other passenger was in the lift, the first fellow creature she had seen face to face for months. Few travelled in these days, for, thanks to the advance of science, the earth was exactly alike all over. Rapid intercourse, from which the previous civilization had hoped so much, had ended by defeating itself. What was the good of going to Peking when it was just like Shrewsbury? Why return to Shrewsbury when it would all be like Peking? Men seldom moved their bodies; all unrest was concentrated in the soul.

The air-ship service was a relic form the former age. It was kept up, because it was easier to keep it up than to stop it or to diminish it, but it now far exceeded the wants of the population. Vessel after vessel would rise form the vomitories of Rye or of Christchurch (I use the antique names), would sail into the crowded sky, and would draw up at the wharves of the south---empty. So nicely adjusted was the system, so independent of meteorology, that the sky, whether calm or cloudy, resembled a vast kaleidoscope whereon the same patterns periodically recurred. The ship on which Vashti sailed started now at sunset, now at dawn. But always, as it passed above Rheas, it would neighbour the ship that served between Helsingfors and the Brazils, and, every third time it surmounted the Alps, the fleet of Palermo would cross its track behind. Night and day, wind and storm, tide and earthquake, impeded man no longer. He had harnessed Leviathan. All the old literature, with its praise of Nature, and its fear of Nature, rang false as the prattle of a child.

Yet as Vashti saw the vast flank of the ship, stained with exposure to the outer air, her horror of direct experience returned. It was not quite like the air-ship in the cinematophote. For one thing it smelt---not strongly or unpleasantly, but it did smell, and with her eyes shut she should have known that a new thing was close to her. Then she had to walk to it from the lift, had to submit to glances form the other passengers. The man in front dropped his Book---no great matter, but it disquieted them all. In the rooms, if the Book was dropped, the floor raised it mechanically, but the gangway to the air-ship was not so prepared, and the sacred volume lay motionless. They stopped---the thing was unforeseen---and the man, instead of picking up his property, felt the muscles of his arm to see how they had failed him. Then some one actually said with direct utterance: ``We shall be late''---and they trooped on board, Vashti treading on the pages as she did so.

Inside, her anxiety increased. The arrangements were old- fashioned and rough. There was even a female attendant, to whom she would have to announce her wants during the voyage. Of course a revolving platform ran the length of the boat, but she was expected to walk from it to her cabin. Some cabins were better than others, and she did not get the best. She thought the attendant had been unfair, and spasms of rage shook her. The glass valves had closed, she could not go back. She saw, at the end of the vestibule, the lift in which she had ascended going quietly up and down, empty. Beneath those corridors of shining tiles were rooms, tier below tier, reaching far into the earth, and in each room there sat a human being, eating, or sleeping, or producing ideas. And buried deep in the hive was her own room. Vashti was afraid.

``O Machine!'' she murmured, and caressed her Book, and was comforted.

Then the sides of the vestibule seemed to melt together, as do the passages that we see in dreams, the lift vanished , the Book that had been dropped slid to the left and vanished, polished tiles rushed by like a stream of water, there was a slight jar, and the air-ship, issuing from its tunnel, soared above the waters of a tropical ocean.

It was night. For a moment she saw the coast of Sumatra edged by the phosphorescence of waves, and crowned by lighthouses, still sending forth their disregarded beams. These also vanished, and only the stars distracted her. They were not motionless, but swayed to and fro above her head, thronging out of one sky-light into another, as if the universe and not the air-ship was careening. And, as often happens on clear nights, they seemed now to be in perspective, now on a plane; now piled tier beyond tier into the infinite heavens, now concealing infinity, a roof limiting for ever the visions of men. In either case they seemed intolerable. ''Are we to travel in the dark?'' called the passengers angrily, and the attendant, who had been careless, generated the light, and pulled down the blinds of pliable metal. When the air-ships had been built, the desire to look direct at things still lingered in the world. Hence the extraordinary number of skylights and windows, and the proportionate discomfort to those who were civilized and refined. Even in Vashti's cabin one star peeped through a flaw in the blind, and after a few hours' uneasy slumber, she was disturbed by an unfamiliar glow, which was the dawn.

Quick as the ship had sped westwards, the earth had rolled eastwards quicker still, and had dragged back Vashti and her companions towards the sun. Science could prolong the night, but only for a little, and those high hopes of neutralizing the earth's diurnal revolution had passed, together with hopes that were possibly higher. To ``keep pace with the sun,'' or even to outstrip it, had been the aim of the civilization preceding this. Racing aeroplanes had been built for the purpose, capable of enormous speed, and steered by the greatest intellects of the epoch. Round the globe they went, round and round, westward, westward, round and round, amidst humanity's applause. In vain. The globe went eastward quicker still, horrible accidents occurred, and the Committee of the Machine, at the time rising into prominence, declared the pursuit illegal, unmechanical, and punishable by Homelessness.

Of Homelessness more will be said later.

Doubtless the Committee was right. Yet the attempt to ``defeat the sun'' aroused the last common interest that our race experienced about the heavenly bodies, or indeed about anything. It was the last time that men were compacted by thinking of a power outside the world. The sun had conquered, yet it was the end of his spiritual dominion. Dawn, midday, twilight, the zodiacal path, touched neither men's lives not their hearts, and science retreated into the ground, to concentrate herself upon problems that she was certain of solving.

So when Vashti found her cabin invaded by a rosy finger of light, she was annoyed, and tried to adjust the blind. But the blind flew up altogether, and she saw through the skylight small pink clouds, swaying against a background of blue, and as the sun crept higher, its radiance entered direct, brimming down the wall, like a golden sea. It rose and fell with the air-ship's motion, just as waves rise and fall, but it advanced steadily, as a tide advances. Unless she was careful, it would strike her face. A spasm of horror shook her and she rang for the attendant. The attendant too was horrified, but she could do nothing; it was not her place to mend the blind. She could only suggest that the lady should change her cabin, which she accordingly prepared to do.

People were almost exactly alike all over the world, but the attendant of the air-ship, perhaps owing to her exceptional duties, had grown a little out of the common. She had often to address passengers with direct speech, and this had given her a certain roughness and originality of manner. When Vashti served away form the sunbeams with a cry, she behaved barbarically---she put out her hand to steady her.

``How dare you!'' exclaimed the passenger. ``You forget yourself!''

The woman was confused, and apologized for not having let her fall. People never touched one another. The custom had become obsolete, owing to the Machine.

``Where are we now?'' asked Vashti haughtily.

``We are over Asia,'' said the attendant, anxious to be polite.

``Asia?''

``You must excuse my common way of speaking. I have got into the habit of calling places over which I pass by their unmechanical names.''

``Oh, I remember Asia. The Mongols came from it.''

``Beneath us, in the open air, stood a city that was once called Simla.''

``Have you ever heard of the Mongols and of the Brisbane school?''

``No.''

``Brisbane also stood in the open air.''

``Those mountains to the right---let me show you them.'' She pushed back a metal blind. The main chain of the Himalayas was revealed. ``They were once called the Roof of the World, those mountains.''

``You must remember that, before the dawn of civilization, they seemed to be an impenetrable wall that touched the stars. It was supposed that no one but the gods could exist above their summits. How we have advanced, thanks to the Machine!''

``How we have advanced, thanks to the Machine!'' said Vashti.

``How we have advanced, thanks to the Machine!'' echoed the passenger who had dropped his Book the night before, and who was standing in the passage.

``And that white stuff in the cracks?---what is it?''

``I have forgotten its name.''

``Cover the window, please. These mountains give me no ideas.''

The northern aspect of the Himalayas was in deep shadow: on the Indian slope the sun had just prevailed. The forests had been destroyed during the literature epoch for the purpose of making newspaper-pulp, but the snows were awakening to their morning glory, and clouds still hung on the breasts of Kinchinjunga. In the plain were seen the ruins of cities, with diminished rivers creeping by their walls, and by the sides of these were sometimes the signs of vomitories, marking the cities of to day. Over the whole prospect air-ships rushed, crossing the inter-crossing with incredible aplomb, and rising nonchalantly when they desired to escape the perturbations of the lower atmosphere and to traverse the Roof of the World.

``We have indeed advanced, thanks to the Machine,'' repeated the attendant, and hid the Himalayas behind a metal blind.

The day dragged wearily forward. The passengers sat each in his cabin, avoiding one another with an almost physical repulsion and longing to be once more under the surface of the earth. There were eight or ten of them, mostly young males, sent out from the public nurseries to inhabit the rooms of those who had died in various parts of the earth. The man who had dropped his Book was on the homeward journey. He had been sent to Sumatra for the purpose of propagating the race. Vashti alone was travelling by her private will.

At midday she took a second glance at the earth. The air- ship was crossing another range of mountains, but she could see little, owing to clouds. Masses of black rock hovered below her, and merged indistinctly into grey. Their shapes were fantastic; one of them resembled a prostrate man.

``No ideas here,'' murmured Vashti, and hid the Caucasus behind a metal blind.

In the evening she looked again. They were crossing a golden sea, in which lay many small islands and one peninsula. She repeated, ``No ideas here,'' and hid Greece behind a metal blind.



\section{The Mending Apparatus}

By a vestibule, by a lift, by a tubular railway, by a platform, by a sliding door---by reversing all the steps of her departure did Vashti arrive at her son's room, which exactly resembled her own. She might well declare that the visit was superfluous. The buttons, the knobs, the reading-desk with the Book, the temperature, the atmosphere, the illumination---all were exactly the same. And if Kuno himself, flesh of her flesh, stood close beside her at last, what profit was there in that? She was too well-bred to shake him by the hand.

Averting her eyes, she spoke as follows:

``Here I am. I have had the most terrible journey and greatly retarded the development of my soul. It is not worth it, Kuno, it is not worth it. My time is too precious. The sunlight almost touched me, and I have met with the rudest people. I can only stop a few minutes. Say what you want to say, and then I must return.''

``I have been threatened with Homelessness,'' said Kuno.

She looked at him now.

``I have been threatened with Homelessness, and I could not tell you such a thing through the Machine.''

Homelessness means death. The victim is exposed to the air, which kills him.

``I have been outside since I spoke to you last. The tremendous thing has happened, and they have discovered me.''

``But why shouldn't you go outside?'' she exclaimed, ``It is perfectly legal, perfectly mechanical, to visit the surface of the earth. I have lately been to a lecture on the sea; there is no objection to that; one simply summons a respirator and gets an Egression-permit. It is not the kind of thing that spiritually minded people do, and I begged you not to do it, but there is no legal objection to it.''

``I did not get an Egression-permit.''

``Then how did you get out?''

``I found out a way of my own.''

The phrase conveyed no meaning to her, and he had to repeat it.

``A way of your own?'' she whispered. ``But that would be wrong.''

``Why?''

The question shocked her beyond measure.

``You are beginning to worship the Machine,'' he said coldly.

``You think it irreligious of me to have found out a way of my own. It was just what the Committee thought, when they threatened me with Homelessness.''

At this she grew angry. ``I worship nothing!'' she cried. ``I am most advanced. I don't think you irreligious, for there is no such thing as religion left. All the fear and the superstition that existed once have been destroyed by the Machine. I only meant that to find out a way of your own was---Besides, there is no new way out.''

``So it is always supposed.''

``Except through the vomitories, for which one must have an Egression-permit, it is impossible to get out. The Book says so.''

``Well, the Book's wrong, for I have been out on my feet.''

For Kuno was possessed of a certain physical strength.

By these days it was a demerit to be muscular. Each infant was examined at birth, and all who promised undue strength were destroyed. Humanitarians may protest, but it would have been no true kindness to let an athlete live; he would never have been happy in that state of life to which the Machine had called him; he would have yearned for trees to climb, rivers to bathe in, meadows and hills against which he might measure his body. Man must be adapted to his surroundings, must he not? In the dawn of the world our weakly must be exposed on Mount Taygetus, in its twilight our strong will suffer euthanasia, that the Machine may progress, that the Machine may progress, that the Machine may progress eternally.

``You know that we have lost the sense of space. We say `space is annihilated', but we have annihilated not space, but the sense thereof. We have lost a part of ourselves. I determined to recover it, and I began by walking up and down the platform of the railway outside my room. Up and down, until I was tired, and so did recapture the meaning of `Near' and `Far'. `Near' is a place to which I can get quickly on my feet, not a place to which the train or the air-ship will take me quickly. `Far' is a place to which I cannot get quickly on my feet; the vomitory is `far', though I could be there in thirty-eight seconds by summoning the train. Man is the measure. That was my first lesson. Man's feet are the measure for distance, his hands are the measure for ownership, his body is the measure for all that is lovable and desirable and strong. Then I went further: it was then that I called to you for the first time, and you would not come.

``This city, as you know, is built deep beneath the surface of the earth, with only the vomitories protruding. Having paced the platform outside my own room, I took the lift to the next platform and paced that also, and so with each in turn, until I came to the topmost, above which begins the earth. All the platforms were exactly alike, and all that I gained by visiting them was to develop my sense of space and my muscles. I think I should have been content with this---it is not a little thing,---but as I walked and brooded, it occurred to me that our cities had been built in the days when men still breathed the outer air, and that there had been ventilation shafts for the workmen. I could think of nothing but these ventilation shafts. Had they been destroyed by all the food-tubes and medicine-tubes and music-tubes that the Machine has evolved lately? Or did traces of them remain? One thing was certain. If I came upon them anywhere, it would be in the railway-tunnels of the topmost storey. Everywhere else, all space was accounted for.%I almost fixed ''not a little thing,--- '', but it seems like Forster was in the habit of using a comma, then an em dash. Sorry if this is wrong!

``I am telling my story quickly, but don't think that I was not a coward or that your answers never depressed me. It is not the proper thing, it is not mechanical, it is not decent to walk along a railway-tunnel. I did not fear that I might tread upon a live rail and be killed. I feared something far more intangible-doing what was not contemplated by the Machine. Then I said to myself, `Man is the measure', and I went, and after many visits I found an opening.

``The tunnels, of course, were lighted. Everything is light, artificial light; darkness is the exception. So when I saw a black gap in the tiles, I knew that it was an exception, and rejoiced. I put in my arm---I could put in no more at first---and waved it round and round in ecstasy. I loosened another tile, and put in my head, and shouted into the darkness: `I am coming, I shall do it yet,' and my voice reverberated down endless passages. I seemed to hear the spirits of those dead workmen who had returned each evening to the starlight and to their wives, and all the generations who had lived in the open air called back to me, `You will do it yet, you are coming.'\,''

He paused, and, absurd as he was, his last words moved her.

For Kuno had lately asked to be a father, and his request had been refused by the Committee. His was not a type that the Machine desired to hand on.

``Then a train passed. It brushed by me, but I thrust my head and arms into the hole. I had done enough for one day, so I crawled back to the platform, went down in the lift, and summoned my bed. Ah what dreams! And again I called you, and again you refused.''

She shook her head and said:

``Don't. Don't talk of these terrible things. You make me miserable. You are throwing civilization away.''

``But I had got back the sense of space and a man cannot rest then. I determined to get in at the hole and climb the shaft. And so I exercised my arms. Day after day I went through ridiculous movements, until my flesh ached, and I could hang by my hands and hold the pillow of my bed outstretched for many minutes. Then I summoned a respirator, and started.

``It was easy at first. The mortar had somehow rotted, and I soon pushed some more tiles in, and clambered after them into the darkness, and the spirits of the dead comforted me. I don't know what I mean by that. I just say what I felt. I felt, for the first time, that a protest had been lodged against corruption, and that even as the dead were comforting me, so I was comforting the unborn. I felt that humanity existed, and that it existed without clothes. How can I possibly explain this? It was naked, humanity seemed naked, and all these tubes and buttons and machineries neither came into the world with us, nor will they follow us out, nor do they matter supremely while we are here. Had I been strong, I would have torn off every garment I had, and gone out into the outer air unswaddled. But this is not for me, nor perhaps for my generation. I climbed with my respirator and my hygienic clothes and my dietetic tabloids! Better thus than not at all.

``There was a ladder, made of some prim\ae val metal. The light from the railway fell upon its lowest rungs, and I saw that it led straight upwards out of the rubble at the bottom of the shaft. Perhaps our ancestors ran up and down it a dozen times daily, in their building. As I climbed, the rough edges cut through my gloves so that my hands bled. The light helped me for a little, and then came darkness and, worse still, silence which pierced my ears like a sword. The Machine hums! Did you know that? Its hum penetrates our blood, and may even guide our thoughts. Who knows! I was getting beyond its power. Then I thought: `This silence means that I am doing wrong.' But I heard voices in the silence, and again they strengthened me.'' He laughed. ``I had need of them. The next moment I cracked my head against something.''

She sighed.

``I had reached one of those pneumatic stoppers that defend us from the outer air. You may have noticed them on the air- ship. Pitch dark, my feet on the rungs of an invisible ladder, my hands cut; I cannot explain how I lived through this part, but the voices still comforted me, and I felt for fastenings. The stopper, I suppose, was about eight feet across. I passed my hand over it as far as I could reach. It was perfectly smooth. I felt it almost to the centre. Not quite to the centre, for my arm was too short. Then the voice said: `Jump. It is worth it. There may be a handle in the centre, and you may catch hold of it and so come to us your own way. And if there is no handle, so that you may fall and are dashed to pieces---it is still worth it: you will still come to us your own way.' So I jumped. There was a handle, and---''

He paused. Tears gathered in his mother's eyes. She knew that he was fated. If he did not die today he would die tomorrow. There was not room for such a person in the world. And with her pity disgust mingled. She was ashamed at having borne such a son, she who had always been so respectable and so full of ideas. Was he really the little boy to whom she had taught the use of his stops and buttons, and to whom she had given his first lessons in the Book? The very hair that disfigured his lip showed that he was reverting to some savage type. On atavism the Machine can have no mercy.

``There was a handle, and I did catch it. I hung tranced over the darkness and heard the hum of these workings as the last whisper in a dying dream. All the things I had cared about and all the people I had spoken to through tubes appeared infinitely little. Meanwhile the handle revolved. My weight had set something in motion and I span slowly, and then---

``I cannot describe it. I was lying with my face to the sunshine. Blood poured from my nose and ears and I heard a tremendous roaring. The stopper, with me clinging to it, had simply been blown out of the earth, and the air that we make down here was escaping through the vent into the air above. It burst up like a fountain. I crawled back to it---for the upper air hurts---and, as it were, I took great sips from the edge. My respirator had flown goodness knows where, my clothes were torn. I just lay with my lips close to the hole, and I sipped until the bleeding stopped. You can imagine nothing so curious. This hollow in the grass---I will speak of it in a minute,---the sun shining into it, not brilliantly but through marbled clouds,---the peace, the nonchalance, the sense of space, and, brushing my cheek, the roaring fountain of our artificial air! Soon I spied my respirator, bobbing up and down in the current high above my head, and higher still were many air-ships. But no one ever looks out of air-ships, and in any case they could not have picked me up. There I was, stranded. The sun shone a little way down the shaft, and revealed the topmost rung of the ladder, but it was hopeless trying to reach it. I should either have been tossed up again by the escape, or else have fallen in, and died. I could only lie on the grass, sipping and sipping, and from time to time glancing around me.

``I knew that I was in Wessex, for I had taken care to go to a lecture on the subject before starting. Wessex lies above the room in which we are talking now. It was once an important state. Its kings held all the southern coast form the Andredswald to Cornwall, while the Wansdyke protected them on the north, running over the high ground. The lecturer was only concerned with the rise of Wessex, so I do not know how long it remained an international power, nor would the knowledge have assisted me. To tell the truth I could do nothing but laugh, during this part. There was I, with a pneumatic stopper by my side and a respirator bobbing over my head, imprisoned, all three of us, in a grass-grown hollow that was edged with fern.''

Then he grew grave again.

``Lucky for me that it was a hollow. For the air began to fall back into it and to fill it as water fills a bowl. I could crawl about. Presently I stood. I breathed a mixture, in which the air that hurts predominated whenever I tried to climb the sides. This was not so bad. I had not lost my tabloids and remained ridiculously cheerful, and as for the Machine, I forgot about it altogether. My one aim now was to get to the top, where the ferns were, and to view whatever objects lay beyond.

``I rushed the slope. The new air was still too bitter for me and I came rolling back, after a momentary vision of something grey. The sun grew very feeble, and I remembered that he was in Scorpio---I had been to a lecture on that too. If the sun is in Scorpio, and you are in Wessex, it means that you must be as quick as you can, or it will get too dark. (This is the first bit of useful information I have ever got from a lecture, and I expect it will be the last.) It made me try frantically to breathe the new air, and to advance as far as I dared out of my pond. The hollow filled so slowly. At times I thought that the fountain played with less vigour. My respirator seemed to dance nearer the earth; the roar was decreasing.''

He broke off.

``I don't think this is interesting you. The rest will interest you even less. There are no ideas in it, and I wish that I had not troubled you to come. We are too different, mother.''

She told him to continue.

``It was evening before I climbed the bank. The sun had very nearly slipped out of the sky by this time, and I could not get a good view. You, who have just crossed the Roof of the World, will not want to hear an account of the little hills that I saw---low colourless hills. But to me they were living and the turf that covered them was a skin, under which their muscles rippled, and I felt that those hills had called with incalculable force to men in the past, and that men had loved them. Now they sleep---perhaps for ever. They commune with humanity in dreams. Happy the man, happy the woman, who awakes the hills of Wessex. For though they sleep, they will never die.''

His voice rose passionately.

``Cannot you see, cannot all you lecturers see, that it is we that are dying, and that down here the only thing that really lives in the Machine? We created the Machine, to do our will, but we cannot make it do our will now. It was robbed us of the sense of space and of the sense of touch, it has blurred every human relation and narrowed down love to a carnal act, it has paralysed our bodies and our wills, and now it compels us to worship it. The Machine develops---but not on our lines. The Machine proceeds---but not to our goal. We only exist as the blood corpuscles that course through its arteries, and if it could work without us, it would let us die. Oh, I have no remedy---or, at least, only one---to tell men again and again that I have seen the hills of Wessex as \AE lfrid saw them when he overthrew the Danes.

``So the sun set. I forgot to mention that a belt of mist lay between my hill and other hills, and that it was the colour of pearl.''

He broke off for the second time.

``Go on,'' said his mother wearily.

He shook his head.

``Go on. Nothing that you say can distress me now. I am hardened.''

``I had meant to tell you the rest, but I cannot: I know that I cannot: good-bye.''

Vashti stood irresolute. All her nerves were tingling with his blasphemies. But she was also inquisitive.

``This is unfair,'' she complained. ``You have called me across the world to hear your story, and hear it I will. Tell me---as briefly as possible, for this is a disastrous waste of time---tell me how you returned to civilization.''

``Oh---that!'' he said, starting. ``You would like to hear about civilization. Certainly. Had I got to where my respirator fell down?''

``No---but I understand everything now. You put on your respirator, and managed to walk along the surface of the earth to a vomitory, and there your conduct was reported to the Central Committee.''

``By no means.''

He passed his hand over his forehead, as if dispelling some strong impression. Then, resuming his narrative, he warmed to it again.

``My respirator fell about sunset. I had mentioned that the fountain seemed feebler, had I not?''

``Yes.''

``About sunset, it let the respirator fall. As I said, I had entirely forgotten about the Machine, and I paid no great attention at the time, being occupied with other things. I had my pool of air, into which I could dip when the outer keenness became intolerable, and which would possibly remain for days, provided that no wind sprang up to disperse it. Not until it was too late did I realize what the stoppage of the escape implied. You see---the gap in the tunnel had been mended; the Mending Apparatus; the Mending Apparatus, was after me.

``One other warning I had, but I neglected it. The sky at night was clearer than it had been in the day, and the moon, which was about half the sky behind the sun, shone into the dell at moments quite brightly. I was in my usual place---on the boundary between the two atmospheres---when I thought I saw something dark move across the bottom of the dell, and vanish into the shaft. In my folly, I ran down. I bent over and listened, and I thought I heard a faint scraping noise in the depths.

``At this---but it was too late---I took alarm. I determined to put on my respirator and to walk right out of the dell. But my respirator had gone. I knew exactly where it had fallen---between the stopper and the aperture---and I could even feel the mark that it had made in the turf. It had gone, and I realized that something evil was at work, and I had better escape to the other air, and, if I must die, die running towards the cloud that had been the colour of a pearl. I never started. Out of the shaft---it is too horrible. A worm, a long white worm, had crawled out of the shaft and glided over the moonlit grass. %website had ''gliding'' instead.

``I screamed. I did everything that I should not have done, I stamped upon the creature instead of flying from it, and it at once curled round the ankle. Then we fought. The worm let me run all over the dell, but edged up my leg as I ran. ``Help!' I cried. (That part is too awful. It belongs to the part that you will never know.) `Help!' I cried. (Why cannot we suffer in silence?) `Help!' I cried. When my feet were wound together, I fell, I was dragged away from the dear ferns and the living hills, and past the great metal stopper (I can tell you this part), and I thought it might save me again if I caught hold of the handle. It also was enwrapped, it also. Oh, the whole dell was full of the things. They were searching it in all directions, they were denuding it, and the white snouts of others peeped out of the hole, ready if needed. Everything that could be moved they brought---brushwood, bundles of fern, everything, and down we all went intertwined into hell. The last things that I saw, ere the stopper closed after us, were certain stars, and I felt that a man of my sort lived in the sky. For I did fight, I fought till the very end, and it was only my head hitting against the ladder that quieted me. I woke up in this room. The worms had vanished. I was surrounded by artificial air, artificial light, artificial peace, and my friends were calling to me down speaking-tubes to know whether I had come across any new ideas lately.''

Here his story ended. Discussion of it was impossible, and Vashti turned to go.

``It will end in Homelessness,'' she said quietly.

``I wish it would,'' retorted Kuno.

``The Machine has been most merciful.''

``I prefer the mercy of God.''

``By that superstitious phrase, do you mean that you could live in the outer air?''

``Yes.''

``Have you ever seen, round the vomitories, the bones of those who were extruded after the Great Rebellion?''

``Yes.''

%``Have you ever seen, round the vomitories, the bones of those who were extruded after the Great Rebellion?''

%``Yes.'' %%I don't think Forster repeated these two lines verbatim: probably an error on the part of someone since him.

``They were left where they perished for our edification. A few crawled away, but they perished, too---who can doubt it? And so with the Homeless of our own day. The surface of the earth supports life no longer.''

``Indeed.''

``Ferns and a little grass may survive, but all higher forms have perished. Has any air-ship detected them?''

``No.''

``Has any lecturer dealt with them?''

``No.''

``Then why this obstinacy?''

``Because I have seen them,'' he exploded.

``Seen what?''

``Because I have seen her in the twilight---because she came to my help when I called---because she, too, was entangled by the worms, and, luckier than I, was killed by one of them piercing her throat.''

He was mad. Vashti departed, nor, in the troubles that followed, did she ever see his face again.

\section{The Homeless}

During the years that followed Kuno's escapade, two important developments took place in the Machine. On the surface they were revolutionary, but in either case men's minds had been prepared beforehand, and they did but express tendencies that were latent already.

The first of these was the abolition of respirators.

Advanced thinkers, like Vashti, had always held it foolish to visit the surface of the earth. Air-ships might be necessary, but what was the good of going out for mere curiosity and crawling along for a mile or two in a terrestrial motor? The habit was vulgar and perhaps faintly improper: it was unproductive of ideas, and had no connection with the habits that really mattered. So respirators were abolished, and with them, of course, the terrestrial motors, and except for a few lecturers, who complained that they were debarred access to their subject-matter, the development was accepted quietly. Those who still wanted to know what the earth was like had after all only to listen to some gramophone, or to look into some cinematophote. And even the lecturers acquiesced when they found that a lecture on the sea was none the less stimulating when compiled out of other lectures that had already been delivered on the same subject. ``Beware of first- hand ideas!'' exclaimed one of the most advanced of them. ``First-hand ideas do not really exist. They are but the physical impressions produced by love and fear, and on this gross foundation who could erect a philosophy? Let your ideas be second-hand, and if possible tenth-hand, for then they will be far removed from that disturbing element---direct observation. Do not learn anything about this subject of mine---the French Revolution. Learn instead what I think that Enicharmon thought Urizen thought Gutch thought Ho-Yung thought Chi-Bo-Sing thought Lafcadio-Hearn thought Carlyle thought Mirabeau said about the French Revolution. Through the medium of these ten great minds, the blood that was shed at Paris and the windows that were broken at Versailles will be clarified to an idea which you may employ most profitably in your daily lives. But be sure that the intermediates are many and varied, for in history one authority exists to counteract another. Urizen must counteract the scepticism of Ho-Yung and Enicharmon, I must myself counteract the impetuosity of Gutch. You who listen to me are in a better position to judge about the French Revolution than I am. Your descendants will be even in a better position than you, for they will learn what you think I think, and yet another intermediate will be added to the chain. And in time''---his voice rose---``there will come a generation that had got beyond facts, beyond impressions, a generation absolutely colourless, a generation

\begin{center} \emph{seraphically free\\*
from taint of personality,} \end{center}

which will see the French Revolution not as it happened, nor as they would like it to have happened, but as it would have happened, had it taken place in the days of the Machine.'' %Source had ''live and fear''; ''love'' makes more sense.

Tremendous applause greeted this lecture, which did but voice a feeling already latent in the minds of men---a feeling that terrestrial facts must be ignored, and that the abolition of respirators was a positive gain. It was even suggested that air-ships should be abolished too. This was not done, because air-ships had somehow worked themselves into the Machine's system. But year by year they were used less, and mentioned less by thoughtful men.

The second great development was the re-establishment of religion.

This, too, had been voiced in the celebrated lecture. No one could mistake the reverent tone in which the peroration had concluded, and it awakened a responsive echo in the heart of each. Those who had long worshipped silently, now began to talk. They described the strange feeling of peace that came over them when they handled the Book of the Machine, the pleasure that it was to repeat certain numerals out of it, however little meaning those numerals conveyed to the outward ear, the ecstasy of touching a button, however unimportant, or of ringing an electric bell, however superfluously.

``The Machine,'' they exclaimed, ``feeds us and clothes us and houses us; through it we speak to one another, through it we see one another, in it we have our being. The Machine is the friend of ideas and the enemy of superstition: the Machine is omnipotent, eternal; blessed is the Machine.'' And before long this allocution was printed on the first page of the Book, and in subsequent editions the ritual swelled into a complicated system of praise and prayer. The word ``religion'' was sedulously avoided, and in theory the Machine was still the creation and the implement of man. But in practice all, save a few retrogrades, worshipped it as divine. Nor was it worshipped in unity. One believer would be chiefly impressed by the blue optic plates, through which he saw other believers; another by the mending apparatus, which sinful Kuno had compared to worms; another by the lifts, another by the Book. And each would pray to this or to that, and ask it to intercede for him with the Machine as a whole. Persecution---that also was present. It did not break out, for reasons that will be set forward shortly. But it was latent, and all who did not accept the minimum known as ``undenominational Mechanism'' lived in danger of Homelessness, which means death, as we know. %I've followed the modern/American convention of capitalizing conjunctions when they're at the start of sentences: this ''. but'' and an earlier ''. and''; hope this is as trivial a change as I imagine.

To attribute these two great developments to the Central Committee, is to take a very narrow view of civilization. The Central Committee announced the developments, it is true, but they were no more the cause of them than were the kings of the imperialistic period the cause of war. Rather did they yield to some invincible pressure, which came no one knew whither, and which, when gratified, was succeeded by some new pressure equally invincible. To such a state of affairs it is convenient to give the name of progress. No one confessed the Machine was out of hand. Year by year it was served with increased efficiency and decreased intelligence. The better a man knew his own duties upon it, the less he understood the duties of his neighbour, and in all the world there was not one who understood the monster as a whole. Those master brains had perished. They had left full directions, it is true, and their successors had each of them mastered a portion of those directions. But Humanity, in its desire for comfort, had over-reached itself. It had exploited the riches of nature too far. Quietly and complacently, it was sinking into decadence, and progress had come to mean the progress of the Machine.

As for Vashti, her life went peacefully forward until the final disaster. She made her room dark and slept; she awoke and made the room light. She lectured and attended lectures. She exchanged ideas with her innumerable friends and believed she was growing more spiritual. At times a friend was granted Euthanasia, and left his or her room for the homelessness that is beyond all human conception. Vashti did not much mind. After an unsuccessful lecture, she would sometimes ask for Euthanasia herself. But the death-rate was not permitted to exceed the birth-rate, and the Machine had hitherto refused it to her.

The troubles began quietly, long before she was conscious of them.

One day she was astonished at receiving a message from her son. They never communicated, having nothing in common, and she had only heard indirectly that he was still alive, and had been transferred from the northern hemisphere, where he had behaved so mischievously, to the southern---indeed, to a room not far from her own.

``Does he want me to visit him?'' she thought. ``Never again, never. And I have not the time.''

No, it was madness of another kind.

He refused to visualize his face upon the blue plate, and speaking out of the darkness with solemnity said:

``The Machine stops.''

``What do you say?''

``The Machine is stopping, I know it, I know the signs.''

She burst into a peal of laughter. He heard her and was angry, and they spoke no more.

``Can you imagine anything more absurd?'' she cried to a friend. ``A man who was my son believes that the Machine is stopping. It would be impious if it was not mad.''

``The Machine is stopping?'' her friend replied. ``What does that mean? The phrase conveys nothing to me.''

``Nor to me.''

``He does not refer, I suppose, to the trouble there has been lately with the music?''

``Oh no, of course not. Let us talk about music.''

``Have you complained to the authorities?''

``Yes, and they say it wants mending, and referred me to the Committee of the Mending Apparatus. I complained of those curious gasping sighs that disfigure the symphonies of the Brisbane school. They sound like someone in pain. The Committee of the Mending Apparatus say that it shall be remedied shortly.'' %Original text had ''some one''; not sure if that was an old-timey bit of syntax, but we now use a single word for the meaning clearly intended here.

Obscurely worried, she resumed her life. For one thing, the defect in the music irritated her. For another thing, she could not forget Kuno's speech. If he had known that the music was out of repair---he could not know it, for he detested music---if he had known that it was wrong, ``the Machine stops'' was exactly the venomous sort of remark he would have made. Of course he had made it at a venture, but the coincidence annoyed her, and she spoke with some petulance to the Committee of the Mending Apparatus.

They replied, as before, that the defect would be set right shortly.

``Shortly! At once!'' she retorted. ``Why should I be worried by imperfect music? Things are always put right at once. If you do not mend it at once, I shall complain to the Central Committee.''

``No personal complaints are received by the Central Committee,'' the Committee of the Mending Apparatus replied.

``Through whom am I to make my complaint, then?''

``Through us.''

``I complain then.''

``Your complaint shall be forwarded in its turn.''

``Have others complained?''

This question was unmechanical, and the Committee of the Mending Apparatus refused to answer it.

``It is too bad!'' she exclaimed to another of her friends.

``There never was such an unfortunate woman as myself. I can never be sure of my music now. It gets worse and worse each time I summon it.''

``What is it?''

``I do not know whether it is inside my head, or inside the wall.''

``Complain, in either case.''

``I have complained, and my complaint will be forwarded in its turn to the Central Committee.''

Time passed, and they resented the defects no longer. The defects had not been remedied, but the human tissues in that latter day had become so subservient, that they readily adapted themselves to every caprice of the Machine. The sigh at the crises of the Brisbane symphony no longer irritated Vashti; she accepted it as part of the melody. The jarring noise, whether in the head or in the wall, was no longer resented by her friend. And so with the mouldy artificial fruit, so with the bath water that began to stink, so with the defective rhymes that the poetry machine had taken to emit. All were bitterly complained of at first, and then acquiesced in and forgotten. Things went from bad to worse unchallenged. %Fixed ''. all''

It was otherwise with the failure of the sleeping apparatus. That was a more serious stoppage. There came a day when over the whole world---in Sumatra, in Wessex, in the innumerable cities of Courland and Brazil---the beds, when summoned by their tired owners, failed to appear. It may seem a ludicrous matter, but from it we may date the collapse of humanity. The Committee responsible for the failure was assailed by complainants, whom it referred, as usual, to the Committee of the Mending Apparatus, who in its turn assured them that their complaints would be forwarded to the Central Committee. But the discontent grew, for mankind was not yet sufficiently adaptable to do without sleeping.

``Some one is meddling with the Machine---'' they began. %Had been ''one of meddling''...this makes much more sense.

``Some one is trying to make himself king, to reintroduce the personal element.''

``Punish that man with Homelessness.''

``To the rescue! Avenge the Machine! Avenge the Machine!''

``War! Kill the man!''

But the Committee of the Mending Apparatus now came forward, and allayed the panic with well-chosen words. It confessed that the Mending Apparatus was itself in need of repair.

The effect of this frank confession was admirable.

``Of course,'' said a famous lecturer---he of the French Revolution, who gilded each new decay with splendour---``of course we shall not press our complaints now. The Mending Apparatus has treated us so well in the past that we all sympathize with it, and will wait patiently for its recovery. In its own good time it will resume its duties. Meanwhile let us do without our beds, our tabloids, our other little wants. Such, I feel sure, would be the wish of the Machine.''

Thousands of miles away his audience applauded. The Machine still linked them. Under the seas, beneath the roots of the mountains, ran the wires through which they saw and heard, the enormous eyes and ears that were their heritage, and the hum of many workings clothed their thoughts in one garment of subserviency. Only the old and the sick remained ungrateful, for it was rumoured that Euthanasia, too, was out of order, and that pain had reappeared among men.

It became difficult to read. A blight entered the atmosphere and dulled its luminosity. At times Vashti could scarcely see across her room. The air, too, was foul. Loud were the complaints, impotent the remedies, heroic the tone of the lecturer as he cried: ``Courage! courage! What matter so long as the Machine goes on ? To it the darkness and the light are one.'' And though things improved again after a time, the old brilliancy was never recaptured, and humanity never recovered from its entrance into twilight. There was an hysterical talk of ``measures,'' of ``provisional dictatorship,'' and the inhabitants of Sumatra were asked to familiarize themselves with the workings of the central power station, the said power station being situated in France. But for the most part panic reigned, and men spent their strength praying to their Books, tangible proofs of the Machine's omnipotence. There were gradations of terror---at times came rumours of hope---the Mending Apparatus was almost mended---the enemies of the Machine had been got under---new ``nerve-centres'' were evolving which would do the work even more magnificently than before. But there came a day when, without the slightest warning, without any previous hint of feebleness, the entire communication-system broke down, all over the world, and the world, as they understood it, ended.

Vashti was lecturing at the time and her earlier remarks had been punctuated with applause. As she proceeded the audience became silent, and at the conclusion there was no sound. Somewhat displeased, she called to a friend who was a specialist in sympathy. No sound: doubtless the friend was sleeping. And so with the next friend whom she tried to summon, and so with the next, until she remembered Kuno's cryptic remark, ``The Machine stops''.

The phrase still conveyed nothing. If Eternity was stopping it would of course be set going shortly.

For example, there was still a little light and air---the atmosphere had improved a few hours previously. There was still the Book, and while there was the Book there was security.

Then she broke down, for with the cessation of activity came an unexpected terror---silence.

She had never known silence, and the coming of it nearly killed her---it did kill many thousands of people outright. Ever since her birth she had been surrounded by the steady hum. It was to the ear what artificial air was to the lungs, and agonizing pains shot across her head. And scarcely knowing what she did, she stumbled forward and pressed the unfamiliar button, the one that opened the door of her cell.

Now the door of the cell worked on a simple hinge of its own. It was not connected with the central power station, dying far away in France. It opened, rousing immoderate hopes in Vashti, for she thought that the Machine had been mended. It opened, and she saw the dim tunnel that curved far away towards freedom. One look, and then she shrank back. For the tunnel was full of people---she was almost the last in that city to have taken alarm.

People at any time repelled her, and these were nightmares from her worst dreams. People were crawling about, people were screaming, whimpering, gasping for breath, touching each other, vanishing in the dark, and ever and anon being pushed off the platform onto the live rail. Some were fighting round the electric bells, trying to summon trains which could not be summoned. Others were yelling for Euthanasia or for respirators, or blaspheming the Machine. Others stood at the doors of their cells fearing, like herself, either to stop in them or to leave them. And behind all the uproar was silence---the silence which is the voice of the earth and of the generations who have gone. %Fixed ''on to the live rail'', which kind of makes sense but doesn't really.

No---it was worse than solitude. She closed the door again and sat down to wait for the end. The disintegration went on, accompanied by horrible cracks and rumbling. The valves that restrained the Medical Apparatus must have weakened, for it ruptured and hung hideously from the ceiling. The floor heaved and fell and flung her from the chair. A tube oozed towards her serpent fashion. And at last the final horror approached---light began to ebb, and she knew that civilization's long day was closing.

She whirled around, praying to be saved from this, at any rate, kissing the Book, pressing button after button. The uproar outside was increasing, and even penetrated the wall. Slowly the brilliancy of her cell was dimmed, the reflections faded from the metal switches. Now she could not see the reading-stand, now not the Book, though she held it in her hand. Light followed the flight of sound, air was following light, and the original void returned to the cavern from which it has so long been excluded. Vashti continued to whirl, like the devotees of an earlier religion, screaming, praying, striking at the buttons with bleeding hands.

It was thus that she opened her prison and escaped---escaped in the spirit: at least so it seems to me, ere my meditation closes. That she escapes in the body---I cannot perceive that. She struck, by chance, the switch that released the door, and the rush of foul air on her skin, the loud throbbing whispers in her ears, told her that she was facing the tunnel again, and that tremendous platform on which she had seen men fighting. They were not fighting now. Only the whispers remained, and the little whimpering groans. They were dying by hundreds out in the dark.

She burst into tears.

Tears answered her.

They wept for humanity, those two, not for themselves. They could not bear that this should be the end. Ere silence was completed their hearts were opened, and they knew what had been important on the earth. Man, the flower of all flesh, the noblest of all creatures visible, man who had once made god in his image, and had mirrored his strength on the constellations, beautiful naked man was dying, strangled in the garments that he had woven. Century after century had he toiled, and here was his reward. Truly the garment had seemed heavenly at first, shot with colours of culture, sewn with the threads of self-denial. And heavenly it had been so long as man could shed it at will and live by the essence that is his soul, and the essence, equally divine, that is his body. The sin against the body---it was for that they wept in chief; the centuries of wrong against the muscles and the nerves, and those five portals by which we can alone apprehend---glozing it over with talk of evolution, until the body was white pap, the home of ideas as colourless, last sloshy stirrings of a spirit that had grasped the stars.

``Where are you?'' she sobbed.

His voice in the darkness said, ``Here.''

``Is there any hope, Kuno?''

``None for us.''

``Where are you?''

She crawled over the bodies of the dead. His blood spurted over her hands.

``Quicker,'' he gasped, ``I am dying---but we touch, we talk, not through the Machine.''

He kissed her.

``We have come back to our own. We die, but we have recaptured life, as it was in Wessex, when \AE lfrid overthrew the Danes. We know what they know outside, they who dwelt in the cloud that is the colour of a pearl.''

``But Kuno, is it true? Are there still men on the surface of the earth? Is this---tunnel, this poisoned darkness---really not the end?''

He replied:

``I have seen them, spoken to them, loved them. They are hiding in the mist and the ferns until our civilization stops. Today they are the Homeless---tomorrow---'' %originally, this had been a much longer dash; struggling to typeset it.

``Oh, tomorrow---some fool will start the Machine again, tomorrow.''

``Never,'' said Kuno, ``never. Humanity has learnt its lesson.''

As he spoke, the whole city was broken like a honeycomb. An air-ship had sailed in through the vomitory into a ruined wharf. It crashed downwards, exploding as it went, rending gallery after gallery with its wings of steel. For a moment they saw the nations of the dead, and, before they joined them, scraps of the untainted sky.

%
\chapter{Scanners Live in Vain}
\thispagestyle{empty}
%{\hspace{0.25in} \includegraphics{./ru_sun.jpg} }
\section*{\huge \center Cordwainer Smith}
\newpage
%
%%On font switching: http://tex.stackexchange.com/questions/47546/using-multiple-font-types

\section{Home}

Martel was angry. He did not even adjust his blood away from anger. He stamped across the room by judgment, not by sight. When he saw the table hit the floor, and could tell by the expression on Luci's face that the table must have made a loud crash, he looked down to see if his leg were broken. It was not. Scanner to the core, he had to scan himself. The action was reflex and automatic. The inventory included his legs, abdomen, Chestbox of instruments, hands, arms, face, and back with the mirror. Only then did Martel go back to being angry. He talked with his voice, even though he knew that his wife hated its blare and preferred to have him write.

``I tell you, I must cranch. I have to cranch. It's my worry, isn't it?''

When Luci answered, he saw only a part of her words as he read her lips: ``Darling . . . you're my husband . . . right to love you . . . dangerous . . . do it . . . dangerous . . . wait. . . .''

He faced her, but put sound in his voice, letting the blare hurt her again: ``I tell you, I am going to cranch.''

Catching her expression, he became rueful and a little tender: ``Can't you understand what it means to me? To get out of this horrible prison in my own head? To be a man again---hearing your voice, smelling smoke? To feel again---to feel my feet on the ground, to feel the air move against my face? Don't you know what it means?''

Her wide-eyed worrisome concern thrust him back into pure annoyance. He read only a few of the words as her lips moved: ``. . . love you . . . your own good . . . don't you think I want you to be human? . . . your own good . . . too much . . . he said . . . they said . . .''

When he roared at her, he realized that his voice must be particularly bad. He knew that the sound hurt her no less than did the words: ``Do you think I wanted you to marry a Scanner? Didn't I tell you we're almost as low as the habermans? We're dead, I tell you. We've got to be dead to do our work. How can anybody go to the Up-and-Out? Can you dream what raw Space is? I warned you. But you married me. All right, you married a man. Please, darling, let me be a man. Let me hear your voice, let me feel the warmth of being alive, of being human. Let me!''

He saw by her look of stricken assent that he had won the argument. He did not use his voice again. Instead, he pulled his tablet up from where it hung against his chest. He wrote on it, using the pointed fingernail of his right forefinger---the talking nail of a Scanner---in quick cleancut script: \texthw{Pls, drlng, whrs crnching wire?}

She pulled the long gold-sheathed wire out of the pocket of her apron. She let its field sphere fall to the carpeted floor. Swiftly, dutifully, with the deft obedience of a Scanner's wife, she wound the Cranching Wire around his head, spirally around his neck and chest. She avoided the instruments set in his chest. She even avoided the radiating scars around the instruments, the stigmata of men who had gone Up and into the Out. Mechanically he lifted a foot as she slipped the wire between his feet. She drew the wire taut. She snapped the small plug into the High-Burden control next to his Heart-Reader. She helped him to sit down, arranging his hands for him, pushing his head back into the cup at the top of the chair. She turned then, full-face toward him, so that he could read her lips easily. Her expression was composed:

``Ready, darling?''

She knelt, scooped up the sphere at the other end of the wire, stood erect calmly, her back to him. He scanned her, and saw nothing in her posture but grief which would have escaped the eye of anyone but a Scanner. She spoke: he could see her chest-muscles moving. She realized that she was not facing him, and turned so that he could see her lips:

``Ready at last?''

He smiled a yes.

She turned her back to him again. (Luci could never bear to watch him go Under-the-Wire.) She tossed the wire-sphere into the air. It caught in the force-field, and hung there. Suddenly it glowed. That was all. All---except for the sudden red stinking roar of coming back to his senses. Coming back, across the wild threshold of pain---

 
\section{Cranched}

When he awakened under the wire, he did not feel as though he had just cranched. Even though it was the second cranching within the week, he felt fit. He lay in the chair. His ears drank in the sound of air touching things in the room. He heard Luci breathing in the next room, where she was hanging up the wire to cool. He smelt the thousand-and-one smells that are in anybody's room: the crisp freshness of the germ-burner, the sour-sweet tang of the humidifier, the odor of the dinner they had just eaten, the smells of clothes, furniture, of people themselves. All these were pure delight. He sang a phrase or two of his favorite song:

 
Here's to the haberman, Up and Out!
Up---oh!---and Out---oh!---Up and Out! . . .

 

He heard Luci chuckle in the next room. He gloated over the sounds of her dress as she swished to the doorway.

She gave him her crooked little smile. ''You sound all right. Are you all right, really?''

Even with this luxury of senses, he scanned. He took the flash-quick inventory which constituted his professional skill. His eyes swept in the news of the instruments. Nothing showed off scale, beyond the Nerve Compression hanging in the edge of Danger. But he could not worry about the Nerve-box. That always came through cranching. You couldn't get under the wire without having it show on the Nerve-box. Some day the box would go to Overload and drop back down to Dead. That was the way a haberman ended. But you couldn't have everything. People who went to the Up-and-Out had to pay the price for Space.

Anyhow, he should worry! He was a Scanner. A good one, and he knew it. If he couldn't scan himself, who could? This cranching wasn't too dangerous. Dangerous, but not too dangerous.

Luci put out her hand and ruffled his hair as if she had been reading his thoughts, instead of just following them: ``But you know you shouldn't have! You shouldn't!''

``But I did!'' He grinned at her.

Her gaiety still forced, she said: ``Come on, darling, let's have a good time. I have almost everything there is in the icebox---all your favorite tastes. And I have two new records just full of smells. I tried them out myself, and even I liked them. And you know me---''

``Which?''

``Which what, you old darling?''

He slipped his hand over her shoulders as he limped out of the room. (He could never go back to feeling the floor beneath his feet, feeling the air against his face, without being bewildered and clumsy. As if cranching was real, and being a haberman was a bad dream. But he was a haberman, and a Scanner.) ``You know what I meant, Luci . . . the smells, which you have. Which one did you like, on the record?''

``Well-l-l,'' said she, judiciously, ``there were some lamb chops that were the strangest things---''

He interrupted: ``What are lambtchots?''

``Wait till you smell them. Then guess. I'll tell you this much. It's a smell hundreds and hundreds of years old. They found about it in the old books.''

``Is a lambtchot a Beast?''

``I won't tell you. You've got to wait,'' she laughed, as she helped him sit down and spread out his tasting dishes before him. He wanted to go back over the dinner first, sampling all the pretty things he had eaten, and savoring them this time with his now-living lips and tongue.

When Luci had found the Music Wire and had thrown its sphere up into the force-field, he reminded her of the new smells. She took out the long glass records and set the first one into a transmitter.

``Now sniff!''

A queer, frightening, exciting smell came over the room. It seemed like nothing in this world, nor like anything from the Up-and-Out. Yet it was familiar. His mouth watered. His pulse beat a little faster; he scanned his Heartbox. (Faster, sure enough.) But that smell, what was it? In mock perplexity, he grabbed her hands, looked into her eyes, and growled:

``Tell me, darling! Tell me, or I'll eat you up!''

``That's just right!''

``What?''

``You're right. It should make you want to eat me. It's meat.''

``Meat. Who?''

``Not a person,'' said she, knowledgeably, ``a Beast. A Beast which people used to eat. A lamb was a small sheep---you've seen sheep out in the Wild, haven't you?---and a chop is part of its middle---here!'' She pointed at her chest.

Martel did not hear her. All his boxes had swung over toward Alarm, some to Danger. He fought against the roar of his own mind, forcing his body into excess excitement. How easy it was to be a Scanner when you really stood outside your own body, haberman-fashion, and looked back into it with your eyes alone. Then you could manage the body, rule it coldly even in the enduring agony of Space. But to realize that you were a body, that this thing was ruling you, that the mind could kick the flesh and send it roaring off into panic! That was bad.

He tried to remember the days before he had gone into the Haberman Device, before he had been cut apart for the Up-and-Out. Had he always been subject to the rush of his emotions from his mind to his body, from his body back to his mind, confounding him so that he couldn't scan? But he hadn't been a Scanner then.

He knew what had hit him. Amid the roar of his own pulse, he knew. In the nightmare of the Up-and-Out, that smell had forced its way through to him, while their ship burned off Venus and the habermans fought the collapsing metal with their bare hands. He had scanned them: all were in Danger. Chestboxes went up to Overload and dropped to Dead all around him as he had moved from man to man, shoving the drifting corpses out of his way as he fought to scan each man in turn, to clamp vises on unnoticed broken legs, to snap the Sleeping Valve on men whose instruments showed that they were hopelessly near Overload. With men trying to work and cursing him for a Scanner while he, professional zeal aroused, fought to do his job and keep them alive in the Great Pain of Space, he had smelled that smell. It had fought its way along his rebuilt nerves, past the Haberman cuts, past all the safeguards of physical and mental discipline. In the wildest hour of tragedy, he had smelled aloud. He remembered it was like a bad cranching, connected with the fury and nightmare all around him. He had even stopped his work to scan himself, fearful that the First Effect might come, breaking past all haberman cuts and ruining him with the Pain of Space. But he had come through. His own instruments stayed and stayed at Danger, without nearing Overload. He had done his job, and won a commendation for it. He had even forgotten the burning ship.

All except the smell.

And here the smell was all over again---the smell of meat-with-fire . . .

Luci looked at him with wifely concern. She obviously thought he had cranched too much, and was about to haberman back. She tried to be cheerful: ``You'd better rest, honey.''

He whispered to her: ``Cut---off---that---smell.''

She did not question his word. She cut the transmitter. She even crossed the room and stepped up the room controls until a small breeze flitted across the floor and drove the smells up to the ceiling.

He rose, tired and stiff. (His instruments were normal, except that Heart was fast and Nerves still hanging on the edge of Danger.) He spoke sadly:

``Forgive me, Luci. I suppose I shouldn't have cranched. Not so soon again. But darling, I have to get out from being a haberman. How can I ever be near you? How can I be a man---not hearing my own voice, not even feeling my own life as it goes through my veins? I love you, darling. Can't I ever be near you?''

Her pride was disciplined and automatic: ``But you're a Scanner!''

``I know I'm a Scanner. But so what?''

She went over the words, like a tale told a thousand times to reassure herself: ``You are the bravest of the brave, the most skillful of the skilled. All Mankind owes most honor to the Scanner, who unites the Earths of mankind. Scanners are the protectors of the habermans. They are the judges in the Up-and-Out. They make men live in the place where men need desperately to die. They are the most honored of Mankind, and even the Chiefs of the Instrumentality are delighted to pay them homage!''

With obstinate sorrow he demurred: ``Luci, we've heard that all before. But does it pay us back---''

``\,`Scanners work for more than pay. They are the strong guards of Mankind.' Don't you remember that?''

``But our lives, Luci. What can you get out of being the wife of a Scanner? Why did you marry me? I'm human only when I cranch. The rest of the time---you know what I am. A machine. A man turned into a machine. A man who has been killed and kept alive for duty. Don't you realize what I miss?''

``Of course, darling, of course---''

He went on: ``Don't you think I remember my childhood? Don't you think I remember what it is to be a man and not a haberman? To walk and feel my feet on the ground? To feel decent clean pain instead of watching my body every minute to see if I'm alive? How will I know if I'm dead? Did you ever think of that, Luci? How will I know if I'm dead?''

She ignored the unreasonableness of his outburst. Pacifyingly, she said: ``Sit down, darling. Let me make you some kind of a drink. You're over-wrought.''

Automatically he scanned. ``No, I'm not! Listen to me. How do you think it feels to be in the Up-and-Out with the crew tied-for-Space all around you? How do you think it feels to watch them sleep? How do you think I like scanning, scanning, scanning month after month, when I can feel the Pain-of-Space beating against every part of my body, trying to get past my Haberman blocks? How do you think I like to wake the men when I have to, and have them hate me for it? Have you ever seen habermans fight---strong men fighting, and neither knowing pain, fighting until one touches Overload? Do you think about that, Luci?'' Triumphantly he added: ``Can you blame me if I cranch, and come back to being a man, just two days a month?''

``I'm not blaming you, darling. Let's enjoy your cranch. Sit down now, and have a drink.''

He was sitting down, resting his face in his hands, while she fixed the drink, using natural fruits out of bottles in addition to the secure alkaloids. He watched her restlessly and pitied her for marrying a Scanner; and then, though it was unjust, resented having to pity her.

Just as she turned to hand him the drink, they both jumped a little when the phone rang. It should not have rung. They had turned it off. It rang again, obviously on the emergency circuit. Stepping ahead of Luci, Martel strode over to the phone and looked into it. Vomact was looking at him.

The custom of Scanners entitled him to be brusque, even with a Senior Scanner, on certain given occasions. This was one.

Before Vomact could speak, Martel spoke two words into the plate, not caring whether the old man could read lips or not:

``Cranching. Busy.''

He cut the switch and went back to Luci.

The phone rang again.

Luci said, gently, ``I can find out what it is, darling. Here, take your drink and sit down.''

``Leave it alone,'' said her husband. ``No one has a right to call when I'm cranching. He knows that. He ought to know that.''

The phone rang again. In a fury, Martel rose and went to the plate. He cut it back on. Vomact was on the screen. Before Martel could speak, Vomact held up his Talking Nail in line with his Heartbox. Martel reverted to discipline:

``Scanner Martel present and waiting, sir.''

The lips moved solemnly: ``Top emergency.''

``Sir, I am under the wire.''

``Top emergency.''

``Sir, don't you understand?'' Martel mouthed his words, so he could be sure that Vomact followed. ``I . . . am . . . under . . . the . . . wire. Unfit . . . for . . . Space!''

Vomact repeated: ``Top emergency. Report to your central Tie-in.''

``But, sir, no emergency like this---''

``Right, Martel. No emergency like this, ever before. Report to Tie-in.'' With a faint glint of kindliness, Vomact added: ``No need to de-cranch. Report as you are.''

This time it was Martel whose phone was cut out. The screen went gray.

He turned to Luci. The temper had gone out of his voice. She came to him. She kissed him, and rumpled his hair. All she could say was,

``I'm sorry.''

She kissed him again, knowing his disappointment. ``Take good care of yourself, darling. I'll wait.''

He scanned, and slipped into his transparent aircoat. At the window he paused, and waved. She called, ``Good luck!'' As the air flowed past him he said to himself,

``This is the first time I've felt flight in---in eleven years. Lord, but it's easy to fly if you can feel yourself live!''

Central Tie-in glowed white and austere far ahead. Martel peered. He saw no glare of incoming ships from the Up-and-Out, no shuddering flare of Space-fire out of control. Everything was quiet, as it should be on an off-duty night.

And yet Vomact had called. He had called an emergency higher than Space. There was no such thing. But Vomact had called it.

 
\section{Emergency}

When Martel got there, he found about half the Scanners present, two dozen or so of them. He lifted the Talking Finger. Most of the Scanners were standing face to face, talking in pairs as they read lips. A few of the old, impatient ones were scribbling on their Tablets and then thrusting the Tablets into other people's faces. All the faces wore the dull dead relaxed look of a haberman. When Martel entered the room, he knew that most of the others laughed in the deep isolated privacy of their own minds, each thinking things it would be useless to express in formal words. It had been a long time since a Scanner showed up at a meeting cranched.

Vomact was not there: probably, thought Martel, he was still on the phone calling others. The light of the phone flashed on and off; the bell rang. Martel felt odd when he realized that of all those present, he was the only one to hear that loud bell. It made him realize why ordinary people did not like to be around groups of habermans or Scanners. Martel looked around for company.

His friend Chang was there, but was busy explaining to some old and testy Scanner that he did not know why Vomact had called. Martel looked further and saw Parizianski. He walked over, threading his way past the others with a dexterity that showed he could feel his feet from the inside, and did not have to watch them. Several of the others stared at him with their dead faces, and tried to smile. But they lacked full muscular control and their faces twisted into horrid masks. (Scanners knew better than to show expression on faces which they could no longer govern. Martel added to himself, I swear I'll never smile unless I'm cranched.)

Parizianski gave him the sign of the talking finger. Looking face to face, he spoke:

``You come here cranched?''

Parizianski could not hear his own voice, so the words roared like the words on a broken and screeching phone; Martel was startled, but knew that the inquiry was well meant. No one could be better-natured than the burly Pole.

``Vomact called. Top emergency.''

``You told him you were cranched?''

``Yes.''

``He still made you come?''

``Yes.''

``Then all this---it is not for Space? You could not go Up-and-Out? You are like ordinary men.''

``That's right.''

``Then why did he call us?'' Some pre-Haberman habit made Parizianski wave his arms in inquiry. The hand struck the back of the old man behind them. The slap could be heard throughout the room, but only Martel heard it. Instinctively, he scanned Parizianski and the old Scanner: they scanned him back. Only then did the old man ask why Martel had scanned him. When Martel explained that he was Under-the-Wire, the old man moved swiftly away to pass on the news that there was a cranched Scanner present at the Tie-in.

Even this minor sensation could not keep the attention of most of the Scanners from the worry about the Top Emergency. One young man, who had scanned his first transit just the year before, dramatically interposed himself between Parizianski and Martel. He dramatically flashed his Tablet at them:

\texthw{Is Vmct mad?}

The older men shook their heads. Martel, remembering that it had not been too long that the young man had been haberman, mitigated the dead solemnity of the denial with a friendly smile. He spoke in a normal voice, saying:

``Vomact is the Senior of Scanners. I am sure that he could not go mad. Would he not see it on his boxes first?''

Martel had to repeat the question, speaking slowly and mouthing his words, before the young Scanner could understand the comment. The young man tried to make his face smile, and twisted it into a comic mask. But he took up his Tablet and scribbled:

\texthw{Yr rght.}

Chang broke away from his friend and came over, his half-Chinese face gleaming in the warm evening. (It's strange, thought Martel, that more Chinese don't become Scanners. Or not so strange, perhaps, if you think that they never fill their quota of habermans. Chinese love good living too much. The ones who do scan are all good ones.) Chang saw that Martel was cranched, and spoke with voice:

``You break precedents. Luci must be angry to lose you?''

``She took it well. Chang, that's strange.''

``What?''

``I'm cranched, and I can hear. Your voice sounds all right. How did you learn to talk like---like an ordinary person?''

``I practiced with soundtracks. Funny you noticed it. I think I am the only Scanner in or between the Earths who can pass for an Ordinary Man. Mirrors and soundtracks. I found out how to act.''

``But you don't . . . ?''

``No. I don't feel, or taste, or hear, or smell things, any more than you do. Talking doesn't do me much good. But I notice that it cheers up the people around me.''

``It would make a difference in the life of Luci.''

Chang nodded sagely. ``My father insisted on it. He said, `You may be proud of being a Scanner. I am sorry you are not a Man. Conceal your defects.' So I tried. I wanted to tell the old boy about the Up-and-Out, and what we did there, but it did not matter. He said, `Airplanes were good enough for Confucius, and they are for me too.' The old humbug! He tries so hard to be a Chinese when he can't even read Old Chinese. But he's got wonderful good sense, and for somebody going on two hundred he certainly gets around.''

Martel smiled at the thought: ``In his airplane?''

Chang smiled back. This discipline of his facial muscles was amazing; a bystander would not think that Chang was a haberman, controlling his eyes, cheeks, and lips by cold intellectual control. The expression had the spontaneity of life. Martel felt a flash of envy for Chang when he looked at the dead cold faces of Parizianski and the others. He knew that he himself looked fine: but why shouldn't he? He was cranched. Turning to Parizianski he said,

``Did you see what Chang said about his father? The old boy uses an airplane.''

Parizianski made motions with his mouth, but the sounds meant nothing. He took up his Tablet and showed it to Martel and Chang:

\texthw{Bzz bzz. Ha ha. Gd ol' boy.}

At that moment, Martel heard steps out in the corridor. He could not help looking toward the door. Other eyes followed the direction of his glance.

Vomact came in.

The group shuffled to attention in four parallel lines. They scanned one another. Numerous hands reached across to adjust the electrochemical controls on Chestboxes which had begun to load up. One Scanner held out a broken finger which his counter-scanner had discovered, and submitted it for treatment and splinting.

Vomact had taken out his Staff of Office. The cube at the top flashed red light through the room, the lines reformed, and all Scanners gave the sign meaning, Present and ready!

Vomact countered with the stance signifying, I am the Senior and take Command.

Talking fingers rose in the counter-gesture, We concur and commit ourselves.

Vomact raised his right arm, dropped the wrist as though it were broken, in a queer searching gesture, meaning: Any men around? Any habermans not tied? All clear for the Scanners?

Alone of all those present, the cranched Martel heard the queer rustle of feet as they all turned completely around without leaving position, looking sharply at one another and flashing their beltlights into the dark corners of the great room. When again they faced Vomact, he made a further sign:

All clear. Follow my words.

Martel noticed that he alone relaxed. The others could not know the meaning of relaxation with the minds blocked off up there in their skulls, connected only with the eyes, and the rest of the body connected with the mind only by controlling non-sensory nerves and the instrument boxes on their chests. Martel realized that, cranched as he was, he expected to hear Vomact's voice: the Senior had been talking for some time. No sound escaped his lips. (Vomact never bothered with sound.)

``. . . and when the first men to go Up and Out went to the Moon, what did they find?''

``Nothing,'' responded the silent chorus of lips.

``Therefore they went further, to Mars and to Venus. The ships went out year by year, but they did not come back until the Year One of Space. Then did a ship come back with the First Effect. Scanners, I ask you, what is the First Effect?''

``No one knows. No one knows.''

``No one will ever know. Too many are the variables. By what do we know the First Effect?''

``By the Great Pain of Space,'' came the chorus.

``And by what further sign?''

``By the need, oh the need for death.''

Vomact again: ``And who stopped the need for death?''

``Henry Haberman conquered the first effect, in the Year 3 of Space.''

``And, Scanners, I ask you, what did he do?''

``He made the habermans.''

``How, O Scanners, are habermans made?''

``They are made with the cuts. The brain is cut from the heart, the lungs. The brain is cut from the ears, the nose. The brain is cut from the mouth, the belly. The brain is cut from desire, and pain. The brain is cut from the world. Save for the eyes. Save for the control of the living flesh.''

``And how, O Scanners, is flesh controlled?''

``By the boxes set in the flesh, the controls set in the chest, the signs made to rule the living body, the signs by which the body lives.''

``How does a haberman live and live?''

``The haberman lives by control of the boxes.''

``Whence come the habermans?''

Martel felt in the coming response a great roar of broken voices echoing through the room as the Scanners, habermans themselves, put sound behind their mouthings:

``Habermans are the scum of Mankind. Habermans are the weak, the cruel, the credulous, and the unfit. Habermans are the sentenced-to-more-than-death. Habermans live in the mind alone. They are killed for Space but they live for Space. They master the ships that connect the Earths. They live in the Great Pain while ordinary men sleep in the cold cold sleep of the transit.''

``Brothers and Scanners, I ask you now: are we habermans or are we not?''

``We are habermans in the flesh. We are cut apart, brain and flesh. We are ready to go to the Up-and-Out. All of us have gone through the Haberman Device.''

``We are habermans then?'' Vomact's eyes flashed and glittered as he asked the ritual question.

Again the chorused answer was accompanied by a roar of voices heard only by Martel: ``Habermans we are, and more, and more. We are the Chosen who are habermans by our own free will. We are the Agents of the Instrumentality of Mankind.''

``What must the others say to us?''

``They must say to us, `You are the bravest of the brave, the most skillful of the skilled. All Mankind owes most honor to the Scanner, who unites the Earths of Mankind. Scanners are the protectors of the habermans. They are the judges in the Up-and-Out. They make men live in the place where men need desperately to die. They are the most honored of mankind, and even the Chiefs of the Instrumentality are delighted to pay them homage!'\,''

Vomact stood more erect: ``What is the secret duty of the Scanner?''

``To obey the Instrumentality only in accordance with Scanner Law.''

``What is the second secret duty of the Scanner?''

``To keep secret our law, and to destroy the acquirers thereof.''

``How to destroy?''

``Twice to the Overload, back and Dead.''

``If habermans die, what the duty then?''

The Scanners all compressed their lips for answer. (Silence was the code.) Martel, who---long familiar with the Code---was a little bored with the proceedings, noticed that Chang was breathing too heavily; he reached over and adjusted Chang's Lung-control and received the thanks of Chang's eyes. Vomact observed the interruption and glared at them both. Martel relaxed, trying to imitate the dead cold stillness of the others. It was so hard to do, when you were cranched.

``If others die, what the duty then?'' asked Vomact.

``Scanners together inform the Instrumentality. Scanners together accept the punishment. Scanners together settle the case.''

``And if the punishment be severe?''

``Then no ships go.''

``And if Scanners not be honored?''

``Then no ships go.''

``And if a Scanner goes unpaid?''

``Then no ships go.''

``And if the Others and the Instrumentality are not in all ways at all times mindful of their proper obligation to the Scanners?''

``Then no ships go.''

``And what, O Scanners, if no ships go?''

``The Earths fall apart. The Wild comes back in. The Old Machines and the Beasts return.''

``What is the known duty of a Scanner?''

``Not to sleep in the Up-and-Out.''

``What is the second duty of a Scanner?''

``To keep forgotten the name of fear.''

``What is the third duty of a Scanner?''

``To use the wire of Eustace Cranch only with care, only with moderation.'' Several pair of eyes looked quickly at Martel before the mouthed chorus went on. ``To cranch only at home, only among friends, only for the purpose of remembering, of relaxing, or of begetting.''

``What is the word of the Scanner?''

``Faithful though surrounded by death.''

``What is the motto of the Scanner?''

``Awake though surrounded by silence.''

``What is the work of the Scanner?''

``Labor even in the heights of the Up-and-Out, loyalty even in the depths of the Earths.''

``How do you know a Scanner?''

``We know ourselves. We are dead though we live. And we Talk with the Tablet and the Nail.''

``What is this Code?''

``This code is the friendly ancient wisdom of Scanners, briefly put that we may be mindful and be cheered by our loyalty to one another.''

At this point the formula should have run: ``We complete the Code. Is there work or word for the Scanners?'' But Vomact said, and he repeated:

``Top Emergency. Top Emergency.''

They gave him the sign, Present and ready!

He said, with every eye straining to follow his lips:

``Some of you know the work of Adam Stone?''

Martel saw lips move, saying: ``The Red Asteroid. The Other who lives at the edge of Space.''

``Adam Stone has gone to the Instrumentality, claiming success for his work. He says that he has found how to Screen Out the Pain of Space. He says that the Up-and-Out can be made safe for ordinary men to work in, to stay awake in. He says that there need be no more Scanners.''

Beltlights flashed on all over the room as Scanners sought the right to speak. Vomact nodded to one of the older men. ``Scanner Smith will speak.''

Smith stepped slowly up into the light, watching his own feet. He turned so that they could see his face. He spoke: ``I say that this is a lie. I say that Stone is a liar. I say that the Instrumentality must not be deceived.''

He paused. Then, in answer to some question from the audience which most of the others did not see, he said:

``I invoke the secret duty of the Scanners.''

Smith raised his right hand for Emergency Attention:

``I say that Stone must die.''

 
\section{Stone}

Martel, still cranched, shuddered as he heard the boos, groans, shouts, squeaks, grunts, and moans which came from the Scanners who forgot noise in their excitement and strove to make their dead bodies talk to one another's deaf ears. Beltlights flashed wildly all over the room. There was a rush for the rostrum and Scanners milled around at the top, vying for attention until Parizianski---by sheer bulk---shoved the others aside and down, and turned to mouth at the group.

``Brother Scanners, I want your eyes.''

The people on the floor kept moving, with their numb bodies jostling one another. Finally Vomact stepped up in front of Parizianski, faced the others, and said:

``Scanners, be Scanners! Give him your eyes.''

Parizianski was not good at public speaking. His lips moved too fast. He waved his hands, which took the eyes of the others away from his lips. Nevertheless, Martel was able to follow most of the message:

``. . . can't do this. Stone may have succeeded. If he has succeeded, it means the end of Scanners. It means the end of habermans, too. None of us will have to fight in the Up-and-Out. We won't have anybody else going Under-the-Wire for a few hours or days of being human. Everybody will be Other. Nobody will have to cranch, never again. Men can be men. The habermans can be killed decently and properly, the way men were killed in the Old Days, without anybody keeping them alive. They won't have to work in the Up-and-Out! There will be no more Great Pain---think of it! No . . . more . . . Great . . . Pain! How do we know that Stone is a liar---'' Lights began flashing directly into his eyes. (The rudest insult of Scanner to Scanner was this.)

Vomact again exercised authority. He stepped in front of Parizianski and said something which the others could not see. Parizianski stepped down from the rostrum. Vomact again spoke:

``I think that some of the Scanners disagree with our Brother Parizianski. I say that the use of the rostrum be suspended till we have had a chance for private discussion. In fifteen minutes I will call the meeting back to order.''

Martel looked around for Vomact when the Senior had rejoined the group on the floor. Finding the Senior, Martel wrote swift script on his Tablet, waiting for a chance to thrust the tablet before the senior's eyes. He had written:

\texthw{Am crnchd. Rspctfly requst prmissn lv now, stnd by fr orders.}

Being cranched did strange things to Martel. Most meetings that he attended seemed formal, hearteningly ceremonial, lighting up the dark inward eternities of habermanhood. When he was not cranched, he noticed his body no more than a marble bust notices its marble pedestal. He had stood with them before. He had stood with them effortless hours, while the long-winded ritual broke through the terrible loneliness behind his eyes, and made him feel that the Scanners, though a confraternity of the damned, were none the less forever honored by the professional requirements of their mutilation.

This time, it was different. Coming cranched, and in full possession of smell-sound-taste-feeling, he reacted more or less as a normal man would. He saw his friends and colleagues as a lot of cruelly driven ghosts, posturing out the meaningless ritual of their indefeasible damnation. What difference did anything make, once you were a haberman? Why all this talk about habermans and Scanners? Habermans were criminals or heretics, and Scanners were gentlemen-volunteers, but they were all in the same fix---except that Scanners were deemed worthy of the short-time return of the Cranching Wire, while habermans were simply disconnected while the ships lay in port and were left suspended until they should be awakened, in some hour of emergency or trouble, to work out another spell of their damnation. It was a rare haberman that you saw on the street---someone of special merit or bravery, allowed to look at mankind from the terrible prison of his own mechanified body. And yet, what Scanner ever pitied a haberman? What Scanner ever honored a haberman except perfunctorily in the line of duty? What had the Scanners, as a guild and a class, ever done for the habermans, except to murder them with a twist of the wrist whenever a haberman, too long beside a Scanner, picked up the tricks of the Scanning trade and learned how to live at his own will, not the will the Scanners imposed? What could the Others, the ordinary men, know of what went on inside the ships? The Others slept in their cylinders, mercifully unconscious until they woke up on whatever other Earth they had consigned themselves to. What could the Others know of the men who had to stay alive within the ship?

What could any Other know of the Up-and-Out? What Other could look at the biting acid beauty of the stars in open Space? What could they tell of the Great Pain, which started quietly in the marrow, like an ache, and proceeded by the fatigue and nausea of each separate nerve cell, brain cell, touchpoint in the body, until life itself became a terrible aching hunger for silence and for death?

He was a Scanner. All right, he was a Scanner. He had been a Scanner from the moment when, wholly normal, he had stood in the sunlight before a Subchief of the Instrumentality, and had sworn:

``I pledge my honor and my life to Mankind. I sacrifice myself willingly for the welfare of Mankind. In accepting the perilous austere Honor, I yield all my rights without exception to the Honorable Chiefs of the Instrumentality and to the Honored Confraternity of Scanners.''

He had pledged.

He had gone into the Haberman Device.

He remembered his Hell. He had not had such a bad one, even though it had seemed to last a hundred-million years, all of them without sleep. He had learned to feel with his eyes. He had learned to see despite the heavy eyeplates set back of his eyeballs to insulate his eyes from the rest of him. He had learned to watch his skin. He still remembered the time he had noticed dampness on his shirt, and had pulled out his scanning mirror only to discover that he had worn a hole in his side by leaning against a vibrating machine. (A thing like that could not happen to him now; he was too adept at reading his own instruments.) He remembered the way that he had gone Up-and-Out, and the way that the Great Pain beat into him, despite the fact that his touch, smell, feeling, and hearing were gone for all ordinary purposes. He remembered killing habermans, and keeping others alive, and standing for months beside the Honorable Scanner-Pilot while neither of them slept. He remembered going ashore on Earth Four, and remembered that he had not enjoyed it, and had realized on that day that there was no reward.

Martel stood among the other Scanners. He hated their awkwardness when they moved, their immobility when they stood still. He hated the queer assortment of smells which their bodies yielded unnoticed. He hated the grunts and groans and squawks which they emitted from their deafness. He hated them, and himself.

How could Luci stand him? He had kept his chestbox reading Danger for weeks while he courted her, carrying the Cranching Wire about with him most illegally, and going direct from one cranch to the other without worrying about the fact that his indicators all crept to the edge of Overload. He had wooed her without thinking of what would happen if she did say, ``Yes.'' She had.

``And they lived happily ever after.'' In Old Books they did, but how could they, in life? He had had eighteen days under-the-wire in the whole of the past year! Yet she had loved him. She still loved him. He knew it. She fretted about him through the long months that he was in the Up-and-Out. She tried to make home mean something to him even when he was haberman, make food pretty when it could not be tasted, make herself lovable when she could not be kissed---or might as well not, since a haberman body meant no more than furniture. Luci was patient.

And now, Adam Stone! (He let his Tablet fade: how could he leave, now?)

God bless Adam Stone!

Martel could not help feeling a little sorry for himself. No longer would the high keen call of duty carry him through two hundred or so years of the Others' time, two million private eternities of his own. He could slouch and relax. He could forget High Space, and let the Up-and-Out be tended by Others. He could cranch as much as he dared. He could be almost normal---almost---for one year or five years or no years. But at least he could stay with Luci. He could go with her into the Wild, where there were Beasts and Old Machines still roving the dark places. Perhaps he would die in the excitement of the hunt, throwing spears at an ancient steel Manshonjagger as it leapt from its lair, or tossing hot spheres at the tribesmen of the Unforgiven who still roamed the Wild. There was still life to live, still a good normal death to die, not the moving of a needle out in the silence and pain of Space!

He had been walking about restlessly. His ears were attuned to the sounds of normal speech, so that he did not feel like watching the mouthings of his brethren. Now they seemed to have come to a decision. Vomact was moving to the rostrum. Martel looked about for Chang, and went to stand beside him. Chang whispered,

``You're as restless as water in mid-air! What's the matter? Decranching?''

They both scanned Martel, but the instruments held steady and showed no sign of the cranch giving out.

The great light flared in its call to attention. Again they formed ranks. Vomact thrust his lean old face into the glare, and spoke:

``Scanners and Brothers, I call for a vote.'' He held himself in the stance which meant: I am the Senior and take Command.

A beltlight flashed in protest.

It was old Henderson. He moved to the rostrum, spoke to Vomact, and---with Vomact's nod of approval---turned full-face to repeat his question:

``Who speaks for the Scanners Out in Space?''

No beltlight or hand answered.

Henderson and Vomact, face to face, conferred for a few moments. Then Henderson faced them again:

``I yield to the Senior in Command. But I do not yield to a Meeting of the Confraternity. There are sixty-eight Scanners, and only forty-seven present, of whom one is cranched and U.D. I have therefore proposed that the Senior in Command assume authority only over an Emergency Committee of the Confraternity, not over a Meeting. Is that agreed and understood by the Honorable Scanners?''

Hands rose in assent.

Chang murmured in Martel's ear, ``Lot of difference that makes! Who can tell the difference between a Meeting and a Committee?'' Martel agreed with the words, but was even more impressed with the way that Chang, while haberman, could control his own voice.

Vomact resumed chairmanship: ``We now vote on the question of Adam Stone.

``First, we can assume that he has not succeeded, and that his claims are lies. We know that from our practical experience as Scanners. The Pain of Space is only part of scanning,'' (But the essential part, the basis of it all, thought Martel.) ``and we can rest assured that Stone cannot solve the problem of Space Discipline.''

``That tripe again,'' whispered Chang, unheard save by Martel.

``The Space Discipline of our Confraternity has kept High Space clean of war and dispute. Sixty-eight disciplined men control all High Space. We are removed by our oath and our haberman status from all Earthly passions.

``Therefore, if Adam Stone has conquered the Pain of Space, so that Others can wreck our confraternity and bring to Space the trouble and ruin which afflicts Earths, I say that Adam Stone is wrong. If Adam Stone succeeds, Scanners live in vain!

``Secondly, if Adam Stone has not conquered the Pain of Space, he will cause great trouble in all the Earths. The Instrumentality and the Subchiefs may not give us as many habermans as we need to operate the ships of Mankind. There will be wild stories, and fewer recruits, and, worst of all, the discipline of the Confraternity may relax if this kind of nonsensical heresy is spread around.

``Therefore, if Adam Stone has succeeded, he threatens the ruin of the Confraternity and should die.

``Therefore, if Adam Stone has not succeeded, he is a liar and a heretic, and should die.''

``I move the death of Adam Stone.''

And Vomact made the sign, The Honorable Scanners are pleased to vote.

 
\section{Help}

Martel grabbed wildly for his beltlight. Chang, guessing ahead, had his light out and ready; its bright beam, voting No, shone straight up at the ceiling. Martel got his light out and threw its beam upward in dissent. Then he looked around. Out of the forty-seven present, he could see only five or six glittering.

Two more lights went on. Vomact stood as erect as a frozen corpse. Vomact's eyes flashed as he stared back and forth over the group, looking for lights. Several more went on. Finally Vomact took the closing stance: May it please the Scanners to count the vote.

Three of the older men went up on the rostrum with Vomact. They looked over the room. (Martel thought: These damned ghosts are voting on the life of a real man, a live man! They have no right to do it. I'll tell the Instrumentality! But he knew that he would not. He thought of Luci and what she might gain by the triumph of Adam Stone: the heart-breaking folly of the vote was then almost too much for Martel to bear.)

All three of the tellers held up their hands in unanimous agreement on the sign of the number: Fifteen against.

Vomact dismissed them with a bow of courtesy. He turned and again took the stance: I am the Senior and take Command.

Marveling at his own daring, Martel flashed his beltlight on. He knew that any one of the bystanders might reach over and twist his Heartbox to Overload for such an act. He felt Chang's hand reaching to catch him by the aircoat. But he eluded Chang's grasp and ran, faster than a Scanner should, to the platform. As he ran, he wondered what appeal to make. He wouldn't get time to say much, and wouldn't be seen by all of them. It was no use talking common sense. Not now. It had to be law.

He jumped up on the rostrum beside Vomact, and took the stance: Scanners, an Illegality!

He violated good custom while speaking, still in the stance: ``A Committee has no right to vote death by a majority vote. It takes two-thirds of a full Meeting.''

He felt Vomact's body lunge behind him, felt himself falling from the rostrum, hitting the floor, hurting his knees and his touch-aware hands. He was helped to his feet. He was scanned. Some Scanner he scarcely knew took his instruments and toned him down.

Immediately Martel felt more calm, more detached, and hated himself for feeling so.

He looked up at the rostrum. Vomact maintained the stance signifying: Order!

The Scanners adjusted their ranks. The two Scanners next to Martel took his arms. He shouted at them, but they looked away, and cut themselves off from communication altogether.

Vomact spoke again when he saw the room was quiet: ``A Scanner came here cranched. Honorable Scanners, I apologize for this. It is not the fault of our great and worthy Scanner and friend, Martel. He came here under orders. I told him not to de-cranch. I hoped to spare him an unnecessary haberman. We all know how happily Martel is married, and we wish his brave experiment well. I like Martel. I respect his judgment. I wanted him here. I knew you wanted him here. But he is cranched. He is in no mood to share in the lofty business of the Scanners. I therefore propose a solution which will meet all the requirements of fairness. I propose that we rule Scanner Martel out of order for his violation of rules. This violation would be inexcusable if Martel were not cranched.

``But at the same time, in all fairness to Martel, I further propose that we deal with the points raised so improperly by our worthy but disqualified brother.''

Vomact gave the sign, The Honorable Scanners are pleased to vote. Martel tried to reach his own beltlight; the dead strong hands held him tightly and he struggled in vain. One lone light shone high: Chang's, no doubt.

Vomact thrust his face into the light again: ``Having the approval of our worthy Scanners and present company for the general proposal, I now move that this Committee declare itself to have the full authority of a Meeting, and that this Committee further make me responsible for all misdeeds which this Committee may enact, to be held answerable before the next full Meeting, but not before any other authority beyond the closed and secret ranks of Scanners.''

Flamboyantly this time, his triumph evident, Vomact assumed the vote stance.

Only a few lights shone: far less, patently, than a minority of one-fourth.

Vomact spoke again. The light shone on his high calm forehead, on his dead relaxed cheekbones. His lean cheeks and chin were half-shadowed, save where the lower light picked up and spotlighted his mouth, cruel even in repose. (Vomact was said to be a descendant of some Ancient Lady who had traversed, in an illegitimate and inexplicable fashion, some hundreds of years of time in a single night. Her name, the Lady Vomact, had passed into legend; but her blood and her archaic lust for mastery lived on in the mute masterful body of her descendant. Martel could believe the old tales as he stared at the rostrum, wondering what untraceable mutation had left the Vomact kith as predators among mankind.) Calling loudly with the movement of his lips, but still without sound, Vomact appealed:

``The Honorable Committee is now pleased to reaffirm the sentence of death issued against the heretic and enemy, Adam Stone.'' Again the vote stance.

Again Chang's light shone lonely in its isolated protest.

Vomact then made his final move:

``I call for the designation of the Senior Scanner present as the manager of the sentence. I call for authorization to him to appoint executioners, one or many, who shall make evident the will and majesty of Scanners. I ask that I be accountable for the deed, and not for the means. The deed is a noble deed, for the protection of Mankind and for the honor of the Scanners; but of the means it must be said that they are to be the best at hand, and no more. Who knows the true way to kill an Other, here on a crowded and watchful Earth? This is no mere matter of discharging a cylindered sleeper, no mere question of upgrading the needle of a haberman. When people die down here, it is not like the Up-and-Out. They die reluctantly. Killing within the Earth is not our usual business, O Brothers and Scanners, as you know well. You must choose me to choose my agent as I see fit. Otherwise the common knowledge will become the common betrayal whereas if I alone know the responsibility, I alone could betray us, and you will not have far to look in case the Instrumentality comes searching.'' (What about the killer you choose? thought Martel. He too will know unless---unless you silence him forever.)

Vomact went into the stance: The Honorable Scanners are pleased to vote.

One light of protest shone; Chang's, again.

Martel imagined that he could see a cruel joyful smile on Vomact's dead face---the smile of a man who knew himself righteous and who found his righteousness upheld and affirmed by militant authority.

Martel tried one last time to come free.

The dead hands held. They were locked like vises until their owners' eyes unlocked them: how else could they hold the piloting month by month?

Martel then shouted: ``Honorable Scanners, this is judicial murder.''

No ear heard him. He was cranched, and alone.

Nonetheless, he shouted again: ``You endanger the Confraternity.''

Nothing happened.

The echo of his voice sounded from one end of the room to the other. No head turned. No eyes met his.

Martel realized that as they paired for talk, the eyes of the Scanners avoided him. He saw that no one desired to watch his speech. He knew that behind the cold faces of his friends there lay compassion or amusement. He knew that they knew him to be cranched---absurd, normal, man-like, temporarily no Scanner. But he knew that in this matter the wisdom of Scanners was nothing. He knew that only a cranched Scanner could feel with his very blood the outrage and anger which deliberate murder would provoke among the Others. He knew that the Confraternity endangered itself, and knew that the most ancient prerogative of law was the monopoly of death. Even the Ancient Nations, in the times of the Wars, before the Wild Machines, before the Beasts, before men went into the Up-and-Out---even the Ancients had known this. How did they say it? Only the State shall kill. The States were gone but the Instrumentality remained, and the Instrumentality could not pardon things which occurred within the Earths but beyond its authority. Death in Space was the business, the right of the Scanners: how could the Instrumentality enforce its law in a place where all men who wakened, wakened only to die in the Great Pain? Wisely did the Instrumentality leave Space to the Scanners, wisely had the Confraternity not meddled inside the Earths. And now the Confraternity itself was going to step forth as an outlaw band, as a gang of rogues as stupid and reckless as the tribes of the Unforgiven!

Martel knew this because he was cranched. Had he been haberman, he would have thought only with his mind, not with his heart and guts and blood. How could the other Scanners know?

Vomact returned for the last time to the Rostrum: The Committee has met and its will shall be done. Verbally he added: ''Senior among you, I ask your loyalty and your silence.''

At that point, the two Scanners let his arms go. Martel rubbed his numb hands, shaking his fingers to get the circulation back into the cold fingertips. With real freedom, he began to think of what he might still do. He scanned himself: the cranching held. He might have an hour, he might have a day. Well, he could go on even if haberman, but it would be inconvenient, having to talk with Finger and Tablet. He looked about for Chang. He saw his friend standing patient and immobile in a quiet corner. Martel moved slowly, so as not to attract any more attention to himself than could be helped. He faced Chang, moved until his face was in the light, and then articulated:

``What are we going to do? You're not going to let them kill Adam Stone, are you? Don't you realize what Stone's work will mean to us, if it succeeds? No more scanning. No more Scanners. No more habermans. No more Pain in the Up-and-Out. I tell you, if the others were all cranched, as I am, they would see it in a human way, not with the narrow crazy logic which they used in the meeting. We've got to stop them. How can we do it? What are we going to do? What does Parizianski think? Who has been chosen?''

``Which question do you want me to answer?''

Martel laughed. (It felt good to laugh, even then; it felt like being a man.) ``Will you help me?''

Chang's eyes flashed across Martel's face as Chang answered: ``No. No. No.''

``You won't help?''

``No.''

``Why not, Chang? Why not?''

``I am a Scanner. The vote has been taken. You would do the same if you were not in this unusual condition.''

``I'm not in an unusual condition. I'm cranched. That merely means that I see things the way that the Others would. I see the stupidity. The recklessness. The selfishness. It is murder.''

``What is murder? Have you not killed? You are not one of the Others. You are a Scanner. You will be sorry for what you are about to do, if you do not watch out.''

``But why did you vote against Vomact then? Didn't you too see what Adam Stone means to all of us? Scanners will live in vain. Thank God for that! Can't you see it?''

``No.''

``But you talk to me, Chang. You are my friend?''

``I talk to you. I am your friend. Why not?''

``But what are you going to do?''

``Nothing, Martel. Nothing.''

``Will you help me?''

``No.''

``Not even to save Stone?''

``Then I will go to Parizianski for help.''

``It will do no good.''

``Why not? He's more human than you, right now.''

``He will not help you, because he has the job. Vomact designated him to kill Adam Stone.''

Martel stopped speaking in mid-movement. He suddenly took the stance: I thank you, Brother, and I depart.

At the window he turned and faced the room. He saw that Vomact's eyes were upon him. He gave the stance, I thank you, Brother, and I depart, and added the flourish of respect which is shown when Seniors are present. Vomact caught the sign, and Martel could see the cruel lips move. He thought he saw the words ``. . . take good care of yourself . . .'' but did not wait to inquire. He stepped backward and dropped out the window.

Once below the window and out of sight, he adjusted his aircoat to maximum speed. He swam lazily in the air, scanning himself thoroughly, and adjusting his adrenal intake down. He then made the movement of release, and felt the cold air rush past his face like running water.

Adam Stone had to be at Chief Downport.

Adam Stone had to be there.

Wouldn't Adam Stone be surprised in the night? Surprised to meet the strangest of beings, the first renegade among Scanners. (Martel suddenly appreciated that it was himself of whom he was thinking. Martel the Traitor to Scanners! That sounded strange and bad. But what of Martel, the Loyal to Mankind? Was that not compensation? And if he won, he won Luci. If he lost, he lost nothing---an unconsidered and expendable haberman. It happened to be himself. But in contrast to the immense reward, to Mankind, to the Confraternity, to Luci, what did that matter?)

Martel thought to himself: ``Adam Stone will have two visitors tonight. Two Scanners, who are the friends of one another.'' He hoped that Parizianski was still his friend.

``And the world,'' he added, ``depends on which of us gets there first.''

Multifaceted in their brightness, the lights of Chief Downport began to shine through the mist ahead. Martel could see the outer towers of the city and glimpsed the phosphorescent periphery which kept back the Wild, whether Beasts, Machines, or the Unforgiven.

Once more Martel invoked the lords of his chance: ``Help me to pass for an Other!''

 
\section{Down}

Within the Downport, Martel had less trouble than he thought. He draped his aircoat over his shoulder so that it concealed the instruments. He took up his scanning mirror, and made up his face from the inside, by adding tone and animation to his blood and nerves until the muscles of his face glowed and the skin gave out a healthy sweat. That way he looked like an ordinary man who had just completed a long night flight.

After straightening out his clothing, and hiding his Tablet within his jacket, he faced the problem of what to do about the Talking Finger. If he kept the nail, it would show him to be a Scanner. He would be respected, but he would be identified. He might be stopped by the guards whom the Instrumentality had undoubtedly set around the person of Adam Stone. If he broke the nail---but he couldn't! No Scanner in the history of the Confraternity had ever willingly broken his nail. That would be Resignation, and there was no such thing. The only way out, was in the Up-and-Out! Martel put his finger to his mouth and bit off the nail. He looked at the now-queer finger, and sighed to himself.

He stepped toward the city gate, slipping his hand into his jacket and running up his muscular strength to four times normal. He started to scan, and then realized that his instruments were masked. Might as well take all the chances at once, he thought.

The watcher stopped him with a searching Wire. The sphere thumped suddenly against Martel's chest.

``Are you a Man?'' said the unseen voice. (Martel knew that as a Scanner in haberman condition, his own field-charge would have illuminated the sphere.)

``I am a Man.'' Martel knew that the timbre of his voice had been good; he hoped that it would not be taken for that of a Manshonjagger or a Beast or an Unforgiven one, who with mimicry sought to enter the cities and ports of Mankind.

``Name, number, rank, purpose, function, time departed.''

``Martel.'' He had to remember his old number, not Scanner 34. ``Sunward 4234, 182nd Year of Space. Rank, rising Subchief.'' That was no lie, but his substantive rank. ``Purpose, personal and lawful within the limits of this city. No function of the Instrumentality. Departed Chief Outport 2019 hours.'' Everything now depended on whether he was believed, or would be checked against Chief Outport.

The voice was flat and routine: ``Time desired within the city.''

Martel used the standard phrase: ``Your Honorable sufferance is requested.''

He stood in the cool night air, waiting. Far above him, through a gap in the mist, he could see the poisonous glittering in the sky of Scanners. The stars are my enemies, he thought: I have mastered the stars but they hate me. Ho, that sounds Ancient! Like a Book. Too much cranching.

The voice returned: ``Sunward 4234 dash 182 rising Subchief Martel, enter the lawful gates of the city. Welcome. Do you desire food, raiment, money, or companionship?'' The voice had no hospitality in it, just business. This was certainly different from entering a city in a Scanner's role! Then the petty officers came out, and threw their beltlights on their fretful faces, and mouthed their words with preposterous deference, shouting against the stone deafness of a Scanner's ears. So that was the way that a Subchief was treated: matter of fact, but not bad. Not bad.

Martel replied: ``I have that which I need, but beg of the city a favor. My friend Adam Stone is here. I desire to see him, on urgent and personal lawful affairs.''

The voice replied: ``Did you have an appointment with Adam Stone?''

``No.''

``The city will find him. What is his number?''

``I have forgotten it.''

``You have forgotten it? Is not Adam Stone a Magnate of the Instrumentality? Are you truly his friend?''

``Truly.'' Martel let a little annoyance creep into his voice. ``Watcher, doubt me and call your Subchief.''

``No doubt implied. Why do you not know the number? This must go into the record,'' added the voice.

``We were friends in childhood. He has crossed the---''Martel started to say ``the Up-and-Out'' and remembered that the phrase was current only among Scanners. ``He has leapt from Earth to Earth, and has just now returned. I knew him well and I seek him out. I have word of his kith. May the Instrumentality protect us!''

``Heard and believed. Adam Stone will be searched.''

At a risk, though a slight one, of having the sphere sound an alarm for non-Man, Martel cut in on his Scanner speaker within his jacket. He saw the trembling needle of light await his words and he started to write on it with his blunt finger. That won't work, he thought, and had a moment's panic until he found his comb, which had a sharp enough tooth to write. He wrote: ``\texthw{Emergency none. Martel Scanner calling Parizianski Scanner.}''

The needle quivered and the reply glowed and faded out: ``Parizianski Scanner on duty and D.C. Calls taken by Scanner Relay.''

Martel cut off his speaker.

Parizianski was somewhere around. Could he have crossed the direct way, right over the city wall, setting off the alert, and invoking official business when the petty officers overtook him in mid-air? Scarcely. That meant that a number of other Scanners must have come in with Parizianski, all of them pretending to be in search of a few of the tenuous pleasures which could be enjoyed by a haberman, such as the sight of the newspictures or the viewing of beautiful women in the Pleasure Gallery. Parizianski was around, but he could not have moved privately, because Scanner Central registered him on duty and recorded his movements city by city.

The voice returned. Puzzlement was expressed in it. ``Adam Stone is found and awakened. He has asked pardon of the Honorable, and says he knows no Martel. Will you see Adam Stone in the morning? The city will bid you welcome.''

Martel ran out of resources. It was hard enough mimicking a man without having to tell lies in the guise of one. Martel could only repeat: ``Tell him I am Martel. The husband of Luci.''

``It will be done.''

Again the silence, and the hostile stars, and the sense that Parizianski was somewhere near and getting nearer; Martel felt his heart beating faster. He stole a glimpse at his chestbox and set his heart down a point. He felt calmer, even though he had not been able to scan with care.

The voice this time was cheerful, as though an annoyance had been settled: ``Adam Stone consents to see you. Enter Chief Downport, and welcome.''

The little sphere dropped noiselessly to the ground and the wire whispered away into the darkness. A bright arc of narrow light rose from the ground in front of Martel and swept through the city to one of the higher towers---apparently a hostel, which Martel had never entered. Martel plucked his aircoat to his chest for ballast, stepped heel-and-toe on the beam, and felt himself whistle through the air to an entrance window which sprang up before him as suddenly as a devouring mouth.

A tower guard stood in the doorway. ``You are awaited, sir. Do you bear weapons, sir?''

``None,'' said Martel, grateful that he was relying on his own strength.

The guard led him past the check-screen. Martel noticed the quick flight of a warning across the screen as his instruments registered and identified him as a Scanner. But the guard had not noticed it.

The guard stopped at a door. ``Adam Stone is armed. He is lawfully armed by authority of the Instrumentality and by the liberty of this city. All those who enter are given warning.''

Martel nodded in understanding at the man, and went in.

Adam Stone was a short man, stout and benign. His gray hair rose stiffly from a low forehead. His whole face was red and merry-looking. He looked like a jolly guide from the Pleasure Gallery, not like a man who had been at the edge of the Up-and-Out, fighting the Great Pain without haberman protection.

He stared at Martel. His look was puzzled, perhaps a little annoyed, but not hostile.

Martel came to the point. ``You do not know me. I lied. My name is Martel, and I mean you no harm. But I lied. I beg the Honorable gift of your hospitality. Remain armed. Direct your weapon against me---''

Stone smiled: ``I am doing so,'' and Martel noticed the small Wirepoint in Stone's capable, plump hand.

``Good. Keep on guard against me. It will give you confidence in what I shall say. But do, I beg you, give us a screen of privacy. I want no casual lookers. This is a matter of life and death.''

``First: whose life and death?'' Stone's face remained calm, his voice even.

``Yours and mine, and the worlds'.''

``You are cryptic but I agree.'' Stone called through the doorway: ``Privacy, please.'' There was a sudden hum, and all the little noises of the night quickly vanished from the air of the room.

Said Adam Stone: ``Sir, who are you? What brings you here?''

``I am Scanner Thirty-four.''

``You a Scanner? I don't believe it.''

For answer, Martel pulled his jacket open, showing his chestbox. Stone looked up at him, amazed. Martel explained:

``I am cranched. Have you never seen it before?''

``Not with men. On animals. Amazing! But---what do you want?''

``The truth. Do you fear me?''

``Not with this,'' said Stone, grasping the Wirepoint. ``But I shall tell you the truth.''

``Is it true that you have conquered the Great Pain?''

Stone hesitated, seeking words for an answer.

``Quick, can you tell me how you have done it, so that I may believe you?''

``I have loaded ships with life.''

``Life?''

``Life. I don't know what the Great Pain is, but I did find that in the experiments, when I sent out masses of animals or plants, the life in the center of the mass lived longest. I built ships---small ones, of course---and sent them out with rabbits, with monkeys---''

``Those are Beasts?''

``Yes. With small Beasts. And the Beasts came back unhurt. They came back because the walls of the ships were filled with life. I tried many kinds, and finally found a sort of life which lives in the waters. Oysters. Oysterbeds. The outermost oysters died in the great pain. The inner ones lived. The passengers were unhurt.''

``But they were Beasts?''

``Not only Beasts. Myself.''

``You!''

``I came through Space alone. Through what you call the Up-and-Out, alone. Awake and sleeping. I am unhurt. If you do not believe me, ask your brother Scanners. Come and see my ship in the morning. I will be glad to see you then, along with your brother Scanners. I am going to demonstrate before the Chiefs of the Instrumentality.''

Martel repeated his question: ``You came here alone?''

Adam Stone grew testy: ``Yes, alone. Go back and check your Scanners' register if you do not believe me. You never put me in a bottle to cross Space.''

Martel's face was radiant. ``I believe you now. It is true. No more Scanners. No more habermans. No more cranching.''

Stone looked significantly toward the door.

Martel did not take the hint. ``I must tell you that---''

``Sir, tell me in the morning. Go enjoy your cranch. Isn't it supposed to be pleasure? Medically I know it well. But not in practice.''

``It is pleasure. It's normality---for a while. But listen. The Scanners have sworn to destroy you, and your work.''

``What!''

``They have met and have voted and sworn. You will make Scanners unnecessary, they say. You will bring the Ancient Wars back to the world, if Scanning is lost and the Scanners live in vain!''

Adam Stone was nervous but kept his wits about him: ``You're a Scanner. Are you going to kill me---or try?''

``No, you fool. I have betrayed the Confraternity. Call guards the moment I escape. Keep guards around you. I will try to intercept the killer.''

Martel saw a blur in the window. Before Stone could turn, the Wirepoint was whipped out of his hand. The blur solidified and took form as Parizianski.

Martel recognized what Parizianski was doing: High speed.

Without thinking of his cranch, he thrust his hand to his chest, set himself up to High speed too. Waves of fire, like the Great Pain, but hotter, flooded over him. He fought to keep his face readable as he stepped in front of Parizianski and gave the sign,

Top Emergency.

Parizianski spoke, while the normally moving body of Stone stepped away from them as slowly as a drifting cloud: ``Get out of my way. I am on a mission.''

``I know it. I stop you here and now. Stop. Stop. Stop. Stone is right.''

Parizianski's lips were barely readable in the haze of pain which flooded Martel. (He thought: God, God, God of the Ancients! Let me hold on! Let me live under Overload just long enough!) Parizianski was saying: ``Get out of my way. By order of the Confraternity, get out of my way!'' And Parizianski gave the sign, Help I demand in the name of my Duty!

Martel choked for breath in the syrup-like air. He tried one last time: ``Parizianski, friend, friend, my friend. Stop. Stop.'' (No Scanner had ever murdered Scanner before.)

Parizianski made the sign: You are unfit for duty, and I will take over.

Martel thought, For the first time in the world! as he reached over and twisted Parizianski's Brainbox up to Overload. Parizianski's eyes glittered in terror and understanding. His body began to drift down toward the floor.

Martel had just strength enough to reach his own Chestbox. As he faded into Haberman or death, he knew not which, he felt his fingers turning on the control of speed, turning down. He tried to speak, to say, ``Get a Scanner, I need help, get a Scanner . . .''

But the darkness rose about him, and the numb silence clasped him.

 
\section{Live}

Martel awakened to see the face of Luci near his own.

He opened his eyes wider, and found that he was hearing---hearing the sound of her happy weeping, the sound of her chest as she caught the air back into her throat.

He spoke weakly: ``Still cranched? Alive?''

Another face swam into the blur beside Luci's. It was Adam Stone. His deep voice rang across immensities of Space before coming to Martel's hearing. Martel tried to read Stone's lips, but could not make them out. He went back to listening to the voice:

``. . . not cranched. Do you understand me? Not cranched!''

Martel tried to say: ``But I can hear! I can feel!'' The others got his sense if not his words.

Adam Stone spoke again:

``You have gone back through the Haberman. I put you back first. I didn't know how it would work in practice, but I had the theory all worked out. You don't think the Instrumentality would waste the Scanners, do you? You go back to normality. We are letting the habermans die as fast as the ships come in. They don't need to live any more. But we are restoring the Scanners. You are the first. Do you understand me? You are the first. Take it easy, now.''

Adam Stone smiled. Dimly behind Stone, Martel thought that he saw the face of one of the Chiefs of the Instrumentality. That face, too, smiled at him, and then both faces disappeared upward and away.

Martel tried to lift his head, to scan himself. He could not. Luci stared at him, calming herself, but with an expression of loving perplexity. She said,

``My darling husband! You're back again, to stay!''

Still, Martel tried to see his box. Finally he swept his hand across his chest with a clumsy motion. There was nothing there. The instruments were gone. He was back to normality but still alive.

In the deep weak peacefulness of his mind, another troubling thought took shape. He tried to write with his finger, the way that Luci wanted him to, but he had neither pointed fingernail nor Scanner's Tablet. He had to use his voice. He summoned up his strength and whispered:

``Scanners?''

``Yes, darling? What is it?''

``Scanners?''

``Scanners. Oh, yes, darling, they're all right. They had to arrest some of them for going into High speed and running away. But the Instrumentality caught them all---all those on the ground---and they're happy now. Do you know, darling,'' she laughed, ``some of them didn't want to be restored to normality. But Stone and the Chiefs persuaded them.''

``Vomact?''

``He's fine, too. He's staying cranched until he can be restored. Do you know, he has arranged for Scanners to take new jobs. You're all Deputy Chiefs for Space. Isn't that nice? But he got himself made Chief for Space. You're all going to be pilots, so that your fraternity and guild can go on. And Chang's getting changed back right now. You'll see him soon.''

Her face turned sad. She looked at him earnestly and said: ``I might as well tell you now. You'll worry otherwise. There has been one accident. Only one. When you and your friend called on Adam Stone, your friend was so happy that he forgot to scan, and he let himself die of Overload.''

``Called on Stone?''

``Yes. Don't you remember? Your friend.''

He still looked surprised, so she said:

``Parizianski.''

%
%\backmatter
%
%\begin{thebibliography}{99}
%\end{thebibliography}
\end{document}
