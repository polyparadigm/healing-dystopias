\chapter{Foreword}

February 2014

I wanted to share two pieces of old-fashioned science fiction in this collection. Both of these works have influenced several generations of speculative fiction writers, and both of them are good stories that have stood the test of time. Both also have the sort of magic it would take to reveal water to a fish: because they were written before I was born, about a time long after my death, they have revealed to me a lot more about my own culture and assumptions than other sorts of writing seem able to. I hope they do the same for the reader.

The two stories in question are  \emph{The Machine Stops} and \emph{Scanners Live In Vain}. I wouldn't quite call either ``dystopian'', but both explore ideas of powerful systems in need of correction.

Each time I read \emph{The Machine Stops}, I see yet another idea or circumstance that is shockingly contemporary for content dating back to 1909. The language is definitely not from the 21\textsuperscript{st} Century, but the lifestyle depicted---a routine of calling up interesting videos and music, mining older cultures for ideas, and pouring it all into remote discussion with ``friends'' one never sees in person, while forgotten automatic systems take care of the necessities of life---looks more current with each passing year. 

But I suspect I glamorize E. M. Forster's clear view of the future, due to a distorted view the past: with apologies to William Gibson, I forget how much of the present was already there---it was just unevenly distributed. The author lived in a time of powered flight, cinematography, and various methods for audio storage and transmission. Dirigibles, a global trade network, and the telegraph system had already resulted in an in\-ter-con\-nect\-ed world. While pervasive access to the intenet would have to wait for at least four score years, the concepts of digitization, error correction, automated repeating, and message switching (not entirely unlike packet switching) had already been commercialized and broadly implemented: global telegraph networks have many of the same needs as global computer networks. Whether Forster was aware of it or not, raster images had also been transmitted electronically, using a device called a Telediagraph. While there weren't TED talks in 1909, there was the Chautauquah circuit, which had similar cultural mechanics even though it relied on physical travel.

This is not to diminish the deft choices of extrapolation that led to ideas like video conferencing and von Neuman-style clanking replicators. But the best science fiction is less an effort to predict the future, than to offer new perspectives on the present. The Machine Stops may resonate with nightmare scenarios about a collapse of global commerce or the de-stabilization of US hegemony, but only because it deals with timeless themes. I would guess that the author examined these themes because they were also apparent in the trade and military empire that had filled the Earth and subdued it in his own time: Edwardian England.

I don't hold romantic ideas about the eternal rightness of some authentic natural way, but studying science and practicing engineering has instilled deep respect, even awe, for the capacity of living systems to heal and to adapt. The systems we build can hold our purposes and unflinchingly drive continued change, but on some level, a lack of adaptability will cause problems when our contraptions encounter limits. 

I can think of several Machines that seem both totalizing and brittle as of this writing, and several confraternities of Scanners who are too absorbed in adjusting the dials and maintaining the \textit{status quo ante} to see the larger picture; I'm sure the reader can, too. What has connected these two stories to my present experience most strongly, though, has been the recent behavior of our intellectual property system. The US Supreme Court ruled that genes cannot be patented, but farmers are still being sued for allowing proprietary pollen to blow into their fields. Patent trolls continue to make good money by stifling innovation. The sharing and adaptation of music, which I think might be even more fundamentally human than inventing new tools or growing food, is being encumbered and co-opted. And worst of all, the ``mending apparatus'' of publishing---journalism---now seems diminished in its capacity to correct itself.

The future of \emph{Scanners Live In Vain} is several decades younger, but its scenes middle-class domesticity served by a whiz-bang transit system sound a lot more old-fashioned. And while both stories include some frank orientalism, Cordwainer Smith shows more mid-century narrow-mindedness in his confident assertions of various sorts of privilege. It isn't much for accurate prediction, but I was just as happy to include it.

I can't think of much from this latter story that came true in the real world, but I see large parallels with other fictional works: notably, The Borg, from \emph{Star Trek}, and the arc of the title character (humoring SCOTUS in their assertion that corporations are persons) of \emph{Monsters Incorporated}. Smith gives us more of an insider's perspective on the sacrifices necessary to keep large systems working, and a more realistic portrayal of the very human work of making such hard decisions. 

Subchiefs and Magnates of the Instrumentality don't have to work as hard as denizens of The Machine to un-do their alienation: the process of re-connecting is mechanized, too. But I find Smith's picture of the process to be more realistic about the internal dynamics: we need social and emotional support to even imagine a different way of life, let alone find the courage to attempt it.

Part of what attracted me to these two stories is their public domain status. The fact that no one can use the legal system to stop anyone from publishing them inspires doubt that publishing them could be profitable, and causes the monetized part of our culture to pay them less attention. I have edited both works to correct what I presume are character-encoding and OCR errors, but it should be noted that I also added chapter names to what had originally been numbered sections in Cordwainer Smith's work.

These two pieces of writing also speak to my struggle to adapt, a struggle I imagine most living things share. Both remind me to pay attention to my gut feelings, and to stay in touch with people who are aware and adaptable. The endings are very different, but I find both to be inspiring. I hope they can have a similar impact, more broadly: Living things heal, and our living traditions are stronger for the criticism of speculative fiction like this.

As Forster wrote elsewhere, ``Only connect!''

Joel Hollingsworth
